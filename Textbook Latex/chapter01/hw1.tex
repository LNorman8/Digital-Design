\section{Exercises}
\label{section:chap01Exercises}

\begin{enumerate}
\item \textbf{ (1 pt. each)} Syllabus:
	\begin{enumerate}
	\item What is the late penalty for homework?
	
	\begin{onlysolution}
	\itshape
	There is a 33\% deduction per day.
	\end{onlysolution}

	\item True or False: Calculators can be used during exams.
	
	\begin{onlysolution}
	\itshape
	You cannot use calculators at my exams.
	\end{onlysolution}
	
	
	\item True of False: University ID is required during exams.
	
	\begin{onlysolution}
	\itshape
	I check ID at the exams.  After I learn 
		your names its not such a big
		deal, but bring it to be safe.
	\end{onlysolution}
	
	\item What is my thesis regarding grades?
	{\begin{onlysolution}
		\fcolorbox{red}{yellow}{lorem ipsum}	
	\end{onlysolution}}	
	\item Bob L. Student has the following grades.  Determine his final
	overall course percentage and grade.

		\begin{tabular}{l|l}
		Component & Percentage \\ \hline \hline
		Homework & $60\%$ \\ \hline
		Exam 1	 & $90\%$ \\ \hline
		Exam 2	 & $80\%$ \\ \hline
		Final	 & $70\%$ \\ 
		\end{tabular}


	\begin{onlysolution}
	\itshape
                \begin{tabular}{l|l|l}
                Component & Percentage & Weight \\ \hline \hline
                Homework & $60\%$    & 60*0.35 = 21\\ \hline
                Exam 1   & $90\%$    & 90*0.20 = 18\\ \hline
                Exam 2   & $80\%$    & 80*0.20 = 16\\ \hline
                Final    & $70\%$    & 70*0.25 = 17.5 \\ \hline
                Total    & $72.5\%$ & C \\
                \end{tabular}
	\end{onlysolution}

	\item How should you prepare for the 43$^{rd}$ lecture?

	\begin{onlysolution}
	\itshape
	 Look over homework problem 8.10, page 165
	 \end{onlysolution}
	 
	\end{enumerate}

\item \textbf{ (1 pt. each)} Convert the following numbers to decimal. 
Show work, or receive 1/2 credit.
	\begin{enumerate}
	\item $100_2$
	\begin{onlysolution}	\itshape $100_2 = 2^2 = 4_{10}$\end{onlysolution}
	
	\item $1000_2$
	\begin{onlysolution}	\itshape $1000_2 = 2^3 = 8_{10}$\end{onlysolution}
	
	\item $10000_2$
	\begin{onlysolution}	\itshape $10000_2 = 2^4 = 16_{10}$\end{onlysolution}
	
	\item $100000_2$
	\begin{onlysolution}	\itshape $100000_2 = 2^5 = 32_{10}$\end{onlysolution}
	
	\item $111111_2$
	\begin{onlysolution}	\itshape $111111_2 = 2^5+2^4+2^3+2^2+2^1+2^0=63_{10}$\end{onlysolution}
	
	\item $1000100101000101_2$
	\begin{onlysolution}	\itshape $1000100101000101_2=2^{15}+2^{11}+2^8+2^6+2^5+2^0=35141_{10}$\end{onlysolution}
	
	\item $3EA_{16}$
	\begin{onlysolution}	\itshape$3EA_{16}=0011 1110 1010 = 2^9+2^8+2^7+2^6+2^5+2^3+2^1=1002_{10}$\\ {\color{blue} $3EA_{16} = 3*16^2 +14*16^1 + 10 * 16^0 = 1002_{10} $}\end{onlysolution}
	
     \end{enumerate}


\item \textbf{ (1 pt. each)} Convert the following number to binary. Show 
work, or receive 1/2 credit.
	\begin{enumerate}
	\item $44_{16}$
	\begin{onlysolution}	\itshape $44_{16} =0100 0100_2$\end{onlysolution}
	
	\item $44_{10}$
	\begin{onlysolution}	\itshape $44_{10} = 32+8 = 2^5+2^3=101100_2$\end{onlysolution}
	
	\item $1023_{10}$
	\begin{onlysolution}	\itshape$1023_{10} = 512+256+128+64+32+16+8+4+2+1=
        2^9+2^8+2^7+2^6+2^5+2^4+2^3+2^2+2^1+2^0=1111111111_2$\end{onlysolution}
        
	\end{enumerate}


\item \textbf{ (1 pt. each)} Convert the following number to hex. Show work, or receive 1/2 credit.
	\begin{enumerate}
	
	\item $101011101_2$
	\begin{onlysolution}	{\color{blue} \itshape$0001\, 0101\, 1101_2 = 15D_{16}$} \end{onlysolution}
	
	\item $77_{10}$
	\begin{onlysolution}	\itshape $77_{10} = 64+8+4+1=2^6+2^3+2^2+2^0=100\, 1101_2=4D_{16}$ \end{onlysolution}
	
	\end{enumerate}


\item \textbf{ (2 pts. each)} Toughies:
	\begin{enumerate}
	\item Convert $123_5$ to base-12
	\begin{onlysolution} 	\itshape
		$123_5 = 1*5^2 + 2*5^1 + 3*5^0 = 25+10+3=38_{10}=
        3*12^1 + 2*12^0 = 32_{12}$ \end{onlysolution}
        
	\item Convert $789_{12}$ to base-5
	\begin{onlysolution}
	\itshape	$789_{12} = 7*12^2+8*12^1+9*12^0=1008+96+9=1113_{10}= \\
        1*5^4+3*5^3 + 4*5^2 + 2*5^1 + 3*5^0 = 13423_5$ \end{onlysolution}
        
	\item What is the largest base-10 quantity that can be represented
	using 5 digits in base 12?

	\begin{onlysolution}
	\itshape $BBBBB_{12} = 11*12^4+11*12^3+11*12^2+11*12^1+11*12^0=248831_{10}$ \end{onlysolution}
	
	\end{enumerate}


\item \textbf{ (1 pt. each)} Perform the following additions, assume a word 
size of four bits. Determine if overflow occurs.
	\begin{enumerate}
	{ \color{blue} % every answer changed, questions caught in the crossfire
	\item $0110_2 + 0101_2$
	\begin{onlysolution}\begin{align*}
	1\0\0\0 &\\
	0110 &\\
 +\:0101 &\\[-2ex]
 \rule{9ex}{1pt}\hspace{-1ex}&\\[-1ex]
	1011& \end{align*}\end{onlysolution}
	
	\item $0010_2 + 0110_2$
	\begin{onlysolution}\begin{align*}
	11\0\0 &\\
    0010 &\\
 +\:0110 &\\[-2ex]
 \rule{9ex}{1pt}\hspace{-1ex}&\\[-1ex]
    1000 &\end{align*}\end{onlysolution}
	
	\item $0111_2 + 0011_2$
	\begin{onlysolution}\begin{align*}
	111\0&\\
    0111&\\
 +\:0011&\\	
 \rule{9ex}{1pt}\hspace{-1ex}&\\[-1ex]
	1010 &\end{align*}\end{onlysolution}
	
	\item $0010_2 + 0101_2$
	\begin{onlysolution}\begin{align*}
	0010&\\
 +\:0101&\\
 \rule{9ex}{1pt}\hspace{-1ex}&\\[-1ex]
    0111 &\end{align*}\end{onlysolution}
	
	\item $0010_2 + 1010_2$
	\begin{onlysolution}	\itshape\begin{align*}
	 1\0\0 &\\
	0010 &\\
 +\:1010 &\\
 \rule{9ex}{1pt}\hspace{-1ex}&\\[-1ex]
    1101 &\end{align*}\end{onlysolution}
	
	\item $0101_2 + 1011_2$
	\begin{onlysolution}\begin{align*}
   1111\0&\\
    0101&\\
 +\:1011&\\
    \hline
    0000&\quad\text{\textit{Overflow}} \end{align*}\end{onlysolution}
	
	\item $0011_2 + 1001_2$
	\begin{onlysolution}	\begin{align*}
     11\0&\\
    0011&\\
 +\:1001&\\
    \hline
    1100&\end{align*}\end{onlysolution} }
	
	\end{enumerate}
\end{enumerate}
