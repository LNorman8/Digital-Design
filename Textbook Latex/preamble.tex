%%------------------------------------------------------
%% Document formatting standards.
%% references labels            figure:<camelCase>
%%                        \tableofcontents:<camelCase>
%%                        section:<dash seperated>
%%                        index:<camelCase>
%%                        example:<camelCase>
%%                        equ:<camelCase>
%%------------------------------------------------------

\usepackage{graphicx}
\usepackage{imakeidx}            % Enables Index functionality
\usepackage{wrapfig}

%% Lots of packages needed for tables
\usepackage{xcolor,colortbl}    % Enable colors in tables
\usepackage{mdframed}

\usepackage{multirow}
\usepackage{hhline}            % allows partial hlines in multirow tables
\usepackage{longtable}            %Enables multipage tables
\usepackage{makecell}            % deal with parboxes inside tables

\usepackage{amssymb}            %% Enables a range of useful  math notation
\usepackage{amsmath}

\usepackage{soul}

\usepackage{tikz}            % Enable drawing simples shapes over text
\usepackage{subcaption}    % Enables tables to not have a caption but to increment the table counter

% This allows conditional formatting of the problem/solution manual
%    \showanswers
%    \hideanswers
\usepackage{probsoln}        % For problems and solutions
\usepackage{dirtree}        % For the body of knowledge

% Table of contents for processes
\usepackage{tocloft}
\usepackage{multicol}

\usepackage{pdflscape}            % enables landscape pages

\usepackage{hyperref}            % Enable hyperlinks to email and URLs
\hypersetup{
    colorlinks=true,
    linkcolor=red,
    menucolor=blue,
    pdftitle={Digital Design, A Datapath and Control Approach}
}

%\newcommand{\ul}{\underline}    %superceeded by soul package
\definecolor{Gray}{gray}{0.85}
\newcolumntype {g} {>{\columncolor{Gray}}c}

% Draws a item list bullet, helps create fake lists in table (tightens up the list)
\newcommand{\tabitem}{~~\llap{\textbullet}~~}

%Adding this before the "\\" in a table line will give you some more room below
\newcommand\B{\rule[-2.5ex]{0pt}{0pt}}         % Bottom strut
%Adding this before the first row item will give you some room above
\newcommand\T{\rule{0pt}{4ex}}       % Top strut

% I want all quotes to be in italics
\newenvironment{itquote}
{
\begin{quote}\itshape}
    {
\end{quote}}

\newcounter{helpfulStuffcounter}
\newcommand{\helpfulStuff}[1]{\refstepcounter{helpfulStuffcounter}\label{#1}}

% Based on the ideas: https://tex.stackexchange.com/questions/241013/mdframed-or-shaded-inside-mdframed

% Credit to: https://tex.stackexchange.com/questions/270598/creating-list-of-examples-using-tocloft-in-memoir-class
\newcounter{processcounter}
\newcommand\listprocessesname{List of Processes}
\newlistof{listofprocesses}{lop}{\listprocessesname}
\newlistentry[chapter]{processcounter}{lop}{0}
\newcommand{\pdescription}[1]{%
\par\noindent\textbf{Process \theprocesscounter~(#1).} }

\newenvironment{process}[1]  {
    \par%
    \addvspace{\baselineskip}
    \refstepcounter{processcounter}
    \addcontentsline{lop}{section}{\protect\numberline{\thechapter.\theprocesscounter} #1}
    \begin{mdframed}[
            frametitle={ \textbf{Process \thechapter.\theprocesscounter}: #1 \vspace{12pt} },
            frametitlerule=false,
            innertopmargin=-0.7em, innerleftmargin=0.4em,
            hidealllines=true, leftline=true,
            linewidth=10pt,
            linecolor=gray!80,
            backgroundcolor=blue!10
        ]
    }{
    \end{mdframed}
}

% Credit to: https://tex.stackexchange.com/questions/270598/creating-list-of-examples-using-tocloft-in-memoir-class
\newcounter{buildingblockcounter}
\newcommand\listbuildingblockname{List of Basic Building Blocks}
\newlistof{listofbuildingblocks}{lobb}{\listbuildingblockname}
\newlistentry[chapter]{buildingblockcounter}{lobb}{0}
\newcommand{\bdescription}[1]{%
\par\noindent\textbf{BuildingBlock \thebuildingblockcounter~(#1).} }

\newenvironment{buildingblock}[1]  {
    \par%
    \addvspace{\baselineskip}
    \refstepcounter{buildingblockcounter}
    \addcontentsline{lobb}{section}{\protect\numberline{\thechapter.\thebuildingblockcounter} #1}
    \begin{mdframed}[
            frametitle={ \textbf{Building Block}: #1 \vspace{12pt} },
            frametitlerule=false,
            innertopmargin=-0.7em, innerleftmargin=0.4em,
            hidealllines=true, leftline=true,
            linewidth=10pt,
            linecolor=gray!80,
            backgroundcolor=green!10
        ]
    }{
    \end{mdframed}
}

% This allows you to fake a figure environment inside
% the example environment
\newcommand{\fakefigure}[1]% #1 = label name
{\refstepcounter{figure}\label{#1}}

\newcommand{\faketable}[1]% #1 = label name
{\refstepcounter{table}\label{#1}}

% Margins need to be wider when you put a kmap in them
\def\changemargin#1#2{\list{}{\rightmargin#2\leftmargin#1}
\item[]}
\let\endchangemargin=\endlist

\newcommand{\SOPmin}{$\text{ SOP}_\text{ min} \ $}
\newcommand{\POSmin}{$\text{ POS}_\text{ min} \ $}
\newcommand{\Tplh}{$\text{ T}_\text{ plh} \ $}
\newcommand{\Tphl}{$\text{ T}_\text{ phl} \ $}
\newcommand{\VCC}{$\text{ V}_\text{ CC} \ $}
\newcommand{\VIH}{$\text{ V}_\text{ IH} \ $}
\newcommand{\VIL}{$\text{ V}_\text{ IL} \ $}
\newcommand{\IOH}{$\text{ I}_\text{ OH} \ $}
\newcommand{\IOL}{$\text{ I}_\text{ OL} \ $}
\newcommand{\TA}{$\text{ T}_\text{ A} \ $}
\newcommand{\VOH}{$\text{ V}_\text{ OH} \ $}
\newcommand{\VOL}{$\text{ V}_\text{ OL} \ $}
\newcommand{\IIH}{$\text{ I}_\text{ IH} \ $}
\newcommand{\IIL}{$\text{ I}_\text{ IL} \ $}
\newcommand{\Ra}{$\text{ R}_\text{ a} \ $}
\newcommand{\Rb}{$\text{ R}_\text{ b} \ $}

% Spacing macros
\newcommand{\1}{\ensuremath{\;\!}} % 2/18 of a quad to provide minimal grouping hints
\newcommand{\0}{\phantom{0}} % width of zero; for formatting addition
\newcommand{\9}{\ensuremath{\,\phantom{0}}} % Zero with a large grouping space after.
