\chapter*{Objectives}
These are the objectives of the lecture
and homework sequence of the course.  They have been arranged
to parallel the chapter/section sequence in the textbook.

\dirtree{%
	.1 \Paste{bok:Digital_Design}.	
	.2 \Paste{bok:Numbering_systems}.
	.3 \Paste{bok:NS_Positional}.
	.4 \Paste{bok:NS_POS_base10}.
	.4 \Paste{bok:NS_POS_base2}.
	.4 \Paste{bok:NS_POS_base16}.
	.3 \Paste{bok:NS_baseConversion}.
	.3 \Paste{bok:NS_Word Size}.
	.3 \Paste{bok:NS_Twos}.
	.2 \Paste{bok:Respresentions}.
	.3 \Paste{bok:REP_Elfs}.
	.3 \Paste{bok:REP_WordStatement}.
	.3 \Paste{bok:REP_TruthTable}.
	.3 \Paste{bok:REP_Symbolic}.
	.3 \Paste{bok:REP_CircuitDiagram}.
	.3 \Paste{bok:REP_HDL}.
	.3 \Paste{bok:REP_Convert}.
	.3 \Paste{bok:REP_MultiOut}.	
	.3 \Paste{bok:REP_Timing}.
	.2 \Paste{bok:LogicMin}.
	.3 \Paste{bok:LM_Kmap}.
	.3 \Paste{bok:LM_MultiOut}.
	.3 \Paste{bok:LM_KmapPos}.
	.3 \Paste{bok:LM_Soft}.
	.2 \Paste{bok:Com_Building_Blocks}.
	.3 \Paste{bok:CC_Dec}.
	.3 \Paste{bok:CC_Mux}.
	.3 \Paste{bok:CC_Adder}.
	.3 \Paste{bok:CC_Compare}.
	.3 \Paste{bok:CC_WireLogic}.
	.3 \Paste{bok:CC_Glue_Combo}.
	.3 \Paste{bok:CC_Combos}.
	.4 \Paste{bok:CC_ArthStatements}.
	.4 \Paste{bok:CC_ConditionalStatements}.	
	.2 \Paste{bok:BasicMemoryElements}.
	.3 \Paste{bok:BME_Char}.
	.3 \Paste{bok:BME_Timing}.
	.3 \Paste{bok:BME_Asynch}.	
	.2 \Paste{bok:BMS_Memory}.	
	.3 \Paste{bok:BMS_Reg}.
	.3 \Paste{bok:BMS_ShitReg}.
	.3 \Paste{bok:BMS_Counter}.
	.3 \Paste{bok:BMS_Ram}.
	.3 \Paste{bok:BMS_RegTran}.	
	.2 \Paste{bok:FSM}.
	.3 \Paste{bok:FSM_Hardware}.
	.3 \Paste{bok:FSM_StateDiagram}.
	.3 \Paste{bok:FSM_OneHot}.
	.3 \Paste{bok:FSM_design}.
	.3 \Paste{bok:FSM_Timing}.
	.2 \Paste{bok:DaC}.
	.3 \Paste{bok:DaC_Architecture}.
	.3 \Paste{bok:DaC_Algorithm}.
	.3 \Paste{bok:DaC_ControWord}.
	.3 \Paste{bok:DaC_Design}.
	.3 \Paste{bok:DaC_Timing}.	
}
\vspace{0.5cm}

These are the objectives of the lab sequence in this class broken down into Verilog  and FPGA.  Verilog
objectives are agnostic to the hardware platform. FPGA objectives are specific to the
hardware and software platformed used to implement the designs laid out in 
these labs.  This organization should enable instructors to quickly identify units 
which may need to be adapted
to run the lab sequence on different hardware/software platform.
\dirtree{%
	.1 \Paste{VER}.
	.2 \Paste{VER:CSA}.
	.2 \Paste{VER:Logic}.
	.2 \Paste{VER:AlwaysCase}.	
	.2 \Paste{VER:AlwaysCaseZ}.	
	.2 \Paste{VER:Vectors}.
	.2 \Paste{VER:Test}.
	.2 \Paste{VER:Generics}.	
	.2 \Paste{VER:Module}.
	.2 \Paste{VER:Instantiating}.
	.2 \Paste{VER:fsm}.	
	.1 \Paste{HDL}.
	.2 \Paste{HDL:Time}.
	.2 \Paste{HDL:Pin}.
	.2 \Paste{HDL:Do}.
	.2 \Paste{HDL:Synthesis}.
}	
		
