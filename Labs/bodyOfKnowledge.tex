\chapter{Objectives}
These are the objectives of the lecture
and homework sequence of the course.  They have been arranged
to parallel the chapter/section sequence in the textbook.

\dirtree{%
	.1 \Copy{bok:Digital_Design}{Digital Design}.	
	.2 \Copy{bok:Numbering_systems}{Numbering Systems}.
	.3 \Copy{bok:NS_Positional}{Positional Numbering Systems}.
	.4 \Copy{bok:NS_POS_base10}{Base 10 - Decimal}.
	.4 \Copy{bok:NS_POS_base2}{Base 2 - Binary}.
	.4 \Copy{bok:NS_POS_base16}{Base 16 - Hexadecimal}.
	.3 \Copy{bok:NS_baseConversion}{Between Bases}.
	.3 \Copy{bok:NS_Word Size}{Word Size}.
	.3 \Copy{bok:NS_Twos}{2's Complement}.
	.2 \Copy{bok:Respresentions}{Representation of Logical Function}.
	.3 \Copy{bok:REP_Elfs}{Elementary Logical Functions}.
	.3 \Copy{bok:REP_WordStatement}{Analyzing a word statement for a logic function}.
	.3 \Copy{bok:REP_TruthTable}{Creating a truth table description for a logic function}.
	.3 \Copy{bok:REP_Symbolic}{Creating a symbolic form for a logic function}.
	.3 \Copy{bok:REP_CircuitDiagram}{Creating a circuit diagram for a logic function}.
	.3 \Copy{bok:REP_HDL}{Creating Hardware Description Language statements for a logic function}.
	.3 \Copy{bok:REP_Convert}{Conversion between two different representations of a logic function}.
	.3 \Copy{bok:REP_MultiOut}{Describing a functions with multiple outputs}.	
	.3 \Copy{bok:REP_Timing}{Timing Diagrams}.
	.2 \Copy{bok:LogicMin}{Logic Minimization}.
	.3 \Copy{bok:LM_Kmap}{Karnaugh Maps (Kmaps)}.
	.3 \Copy{bok:LM_MultiOut}{Kmaps for circuits with multiple outputs}.
	.3 \Copy{bok:LM_KmapPos}{Kmaps to find POSmin}.
	.3 \Copy{bok:LM_Soft}{Using logic minimization software to describe a logic function}
	.2 \Copy{bok:Com_Building_Blocks}{Combination Logic Building Blocks}.
	.3 \Copy{bok:CC_Dec}{Decoder}.
	.3 \Copy{bok:CC_Mux}{Multiplexers}.
	.3 \Copy{bok:CC_Adder}{Adders}.
	.3 \Copy{bok:CC_Compare}{Comparators}.
	.3 \Copy{bok:CC_WireLogic}{Wire Logic}.
	.3 \Copy{bok:CC_Combos}{Combination}.
	.4 \Copy{bok:CC_ArthStatements}{Arithmetic Statements}.
	.4 \Copy{bok:CC_ConditionalStatements}{Conditional Statements}.	
	.2 \Copy{bok:BasicMemoryElements}{Basic Memory Elements}.
	.3 \Copy{bok:BME_Char}{Characteristics}.
	.3 \Copy{bok:BME_Timing}{Timing}.
	.3 \Copy{bok:BME_Asynch}{Asynchronous set reset}.	
	.2 \Copy{bok:BMS_Memory}{Sequential Logic Building Blocks}.	
	.3 \Copy{bok:BMS_Reg}{Analysis, design and use of a register in a digital design}.
	.3 \Copy{bok:BMS_ShitReg}{Shift Register}.
	.3 \Copy{bok:BMS_Counter}{Counter}.
	.3 \Copy{bok:BMS_Ram}{Static RAM}.
	.3 \Copy{bok:BMS_RegTran}{Register Transfer}.	
	.2 \Copy{bok:FSM}{Finite State Machines}.
	.3 \Copy{bok:FSM_Hardware}{Hardware organization of a finite state machine}.
	.3 \Copy{bok:FSM_StateDiagram}{State diagram for a finite state machine}.
	.3 \Copy{bok:FSM_OneHot}{One's Hot Encoding}.
	.3 \Copy{bok:FSM_design}{Design}.
	.3 \Copy{bok:FSM_Timing}{Timing}.
	.2 \Copy{bok:DaC}{Datapath and Control}.
	.3 \Copy{bok:DaC_Architecture}{Datapath and Control Architecture}.
	.3 \Copy{bok:DaC_Algorithm}{Algorithmic Language}.
	.3 \Copy{bok:DaC_Design}{Design}.
	.3 \Copy{bok:DaC_Timing}{Timing}.	
}
\vspace{0.5cm}

These are the objectives of the lab sequence in this class broken down into Verilog  and FPGA.  Verilog
objectives are agnostic to the hardware platform. FPGA objectives are specific to the
hardware and software platformed used to implement the designs laid out in 
these labs.  This organization should enable instructors to quickly identify units 
which may need to be adapted
to run the lab sequence on different hardware/software platform.
\dirtree{%
	.1 \Copy{VER}{Verilog }.
	.2 \Copy{VER:CSA}{Writing concurrent signal assignment statements for a logic function}.
	.2 \Copy{VER:Logic}{Writing a Verilog statement using primitive logic operations}.
	.2 \Copy{VER:AlwaysCase}{Writing a Verilog statement using an Always/Case statement}.	
	.2 \Copy{VER:Vectors}{Creating a Verilog statement that uses vectors}.
	.2 \Copy{VER:Test}{Analyzing and designing a Verilog testbench}.
	.2 \Copy{VER:Generics}{Definition and instantiation of Verilog generic modules}.	
	.2 \Copy{VER:Module}{Definition o Verilog modules}.
	.2 \Copy{VER:Instantiating}{Instantiation of Verilog Modules}.
	.2 \Copy{VER:fsm}{Definition of Finite State Machines in Verilog}.	
	.1 \Copy{HDL}{Hardware and Software Specifics}.
	.2 \Copy{HDL:Time}{Creating a simulation timing diagram for a module}.
	.2 \Copy{HDL:Pin}{Creating a pin assignment for a module}.
	.2 \Copy{HDL:Do}{Creating a Do file to automate waveform setup}.
	.2 \Copy{HDL:Synthesis}{Synthesizing a module on the FPGA development board}.
}	
		
