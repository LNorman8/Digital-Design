\begin{enumerate}
\item {\bf (1 pt. each)} Syllabus:
	\begin{enumerate}
	\item What is the late penalty for homework?
	
	\begin{solution}{There is a 33\% deduction per day.}\end{solution}
	\item True or False: Calculators can be used during exams.
	
	\begin{solution}{You cannot use calculators at my exams.}\end{solution}
	\item True of False: University ID is required during exams.
	
	\begin{solution}{I check ID at the exams.  After I learn 
		your names its not such a big
		deal, but bring it to be safe.}\end{solution}
	\item What is my thesis regarding grades?
	\item Bob L. Student has the following grades.  Determine his final
	overall course percentage and grade.

		\begin{tabular}{l|l}
		Component & Percentage \\ \hline \hline
		Homework & $60\%$ \\ \hline
		Exam 1	 & $90\%$ \\ \hline
		Exam 2	 & $80\%$ \\ \hline
		Final	 & $70\%$ \\ 
		\end{tabular}

	\begin{solution}{
                \begin{tabular}{l|l|l}
                Component & Percentage & Weight \\ \hline \hline
                Homework & $60\%$    & 60*0.35 = 21\\ \hline
                Exam 1   & $90\%$    & 90*0.20 = 18\\ \hline
                Exam 2   & $80\%$    & 80*0.20 = 16\\ \hline
                Final    & $70\%$    & 70*0.25 = 17.5 \\ \hline
                Total    & $72.5\%$ & C \\
                \end{tabular}
}\end{solution}

	\item How should you prepare for the 43$^{rd}$ lecture?

	\begin{solution}{ Look over homework problem 8.10, page 165}\end{solution}
	\end{enumerate}

\item {\bf (1 pt. each)} Convert the following numbers to decimal. 
Show work, or receive 1/2 credit.
	\begin{enumerate}
	\item $100_2$

	\begin{solution}{ $100_2 = 2^2 = 4_{10}$}\end{solution}
	\item $1000_2$

	\begin{solution}{ $1000_2 = 2^3 = 8_{10}$}\end{solution}
	\item $10000_2$

	\begin{solution}{ $10000_2 = 2^4 = 16_{10}$}\end{solution}
	\item $100000_2$

	\begin{solution}{ $100000_2 = 2^5 = 32_{10}$}\end{solution}
	\item $111111_2$

	\begin{solution}{ $111111_2 = 2^5+2^4+2^3+2^2+2^1+2^0=63_{10}$}\end{solution}
	\item $1000100101000101_2$

	\begin{solution}{ $1000100101000101_2=2^{15}+2^{11}+2^8+2^6+2^5+2^0=35141_{10}$}\end{solution}
	\item $3EA_{16}$

	\begin{solution}{ $3EA_{16}=0011 1110 1010 = 2^9+2^8+2^7+2^6+2^5+2^3+2^1=1002_{10}$}\end{solution}
        \end{enumerate}


\item {\bf (1 pt. each)} Convert the following number to binary. Show 
work, or receive 1/2 credit.
	\begin{enumerate}
	\item $44_{16}$

	\begin{solution}{ $44_{16} =0100 0100_2$}\end{solution}
	\item $44_{10}$

	\begin{solution}{ $44_{10} = 32+8 = 2^5+2^3=101100_2$}\end{solution}
	\item $1023_{10}$

	\begin{solution}{ $1023_{10} = 512+256+128+64+32+16+8+4+2+1=
        2^9+2^8+2^7+2^6+2^5+2^4+2^3+2^2+2^1+2^0=1111111111_2$}\end{solution}
	\end{enumerate}


\item {\bf (1 pt. each)} Convert the following number to hex. Show 
work, or receive 1/2 credit.
	\begin{enumerate}
	\item $101011101_2$

	\begin{solution}{ $1 0101 1101_2 = 15D_{16}$ }\end{solution}
	\item $77_{10}$

	\begin{solution}{ $77_{10} = 64+8+4+1=2^6+2^3+2^2+2^0=100 1101_2=4D_{16}$ }\end{solution}
	\end{enumerate}


\item {\bf (2 pts. each)} Toughies:
	\begin{enumerate}
	\item Convert $123_5$ to base-12

	\begin{solution}{       $123_5 = 1*5^2 + 2*5^1 + 3*5^0 = 25+10+3=38_{10}=
        3*12^1 + 2*12^0 = 32_{12}$ }\end{solution}
	\item Convert $789_{12}$ to base-5

	\begin{solution}{	$789_{12} = 7*12^2+8*12^1+9*12^0=1008+96+9=1113_{10}= \\
        1*5^4+3*5^3 + 4*5^2 + 2*5^1 + 3*5^0 = 13423_5$ }\end{solution}
	\item What is the largest base-10 quantity that can be represented
	using 5 digits in base 12?

	\begin{solution}{ $BBBBB_{12} = 11*12^4+11*12^3+11*12^2+11*12^1+11*12^0=248831_{10}$ }\end{solution}
	\end{enumerate}


\item {\bf (1 pt. each)} Perform the following additions, assume a word 
size of four bits. Determine if overflow occurs.
	\begin{enumerate}
	\item $0110_2 + 0101_2$

	\begin{solution}{0110 + 0101 = 1011}\end{solution}
	\item $0010_2 + 0110_2$

	\begin{solution}{0010 + 0110 = 1000}\end{solution}
	\item $0111_2 + 0011_2$

	\begin{solution}{0111 + 0011 = 1010}\end{solution}
	\item $0010_2 + 0101_2$

	\begin{solution}{0010 + 0101 = 0111}\end{solution}
	\item $0010_2 + 1010_2$

	\begin{solution}{0010 + 1010 = 1100}\end{solution}
	\item $0101_2 + 1011_2$

	\begin{solution}{0101 + 1011 = 10000 overflow}\end{solution}
	\item $0011_2 + 1001_2$

	\begin{solution}{0011 + 1001 = 1100}\end{solution}
	\end{enumerate}
\end{enumerate}
