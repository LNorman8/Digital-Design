\chapter{Introduction}

\section{The Design Process}
%% Here we talk about transfroming an idea into a working system.
%% contrast against analysis which is the process of taking a 
%% system and determining its constituent parts.
\pagebreak
.

\section{Digital Systems}
%% In this section I will discuss what it means for something to be
%% a system.  Basically, anything that transforms an input into a
%% output, similar to a function or a relation.  The concept of digital
%% and analog systems is introduced.  The realization that we can
%% enumerate every combination of possible inputs should be mentioned.
.\pagebreak

\section{Realizations}
%% the design process requires that we transform and idea into a working
%% system.  There are a varity of media with which we can realize the
%% final system.  

\subsection{Discrete Components}
%% Digital systems can be realized using a set of chips, each chip 
%% realizing some standard component.  74xx
.\pagebreak

\subsection{PALs and CPLDs}
.\pagebreak

\subsection{FPGA}
%% Discrete realizations are going by the way-side in favor of FPGA
%% based realization.  the advantage is that the wiring of the components
%% is completely accomplished in software.`
\pagebreak
.
\subsection{VLSI}
%% The disadvantage of the two preceeding media are that they require
%% a large amount of space and are slow.  These two problems are solved
%% when a VLSI circuit is created.  The drawback is the engineering time
%% and cost required to design, test, and fabricate a custom chip.
\pagebreak
.
\section{Exercises}
