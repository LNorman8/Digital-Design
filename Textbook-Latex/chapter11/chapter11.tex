\chapter{Optimizing Finite State Machines}
\label{chapter:Optimize Finite State Machines}
\graphicspath{ {./chapter11/Fig} }

The primary purpose of digital design is to construct a circuit that
actualizes some specified word statement.  As an engineer
your secondary goal should be to realize this design in the
most efficient way possible.  There are two main criterion for
efficient, time and space.  Time refers to the speed of the
circuit and is mainly determined by the design
and the propagation delay of the devices.  In
this chapter we will concentrate on the space aspect of
efficiency; minimizing the number of circuit elements required
to realize the design.

Minimizing the number of components in a digital design is,
in general, a very hard problem.  We have already considered
the problem of minimizing combinational logic.  We now turn
our attention to sequential circuit.  There are two optimizations
that can be performed to a FSM; state minimization and state
assignment.

\section{State Minimization}
The purpose of state minimization is to reduce the number
of states in the state table yielding a \textit{ minimal state
table}.  For a FSM designed with a dense
encoding of the states this optimization will, in general, have
very little effect. Since a FSM with $N$ states has
$\lceil \log_2(N) \rceil$ flip flops then the number of states
would have to be decreased by a half on order to save a single
flip flop.  However, in FSM designed using a one's hot encoding,
each state eliminates one flip flop.

The benefit of such a minimization is obvious, reduce the
number of flip flops required for the FSM.  In addition, it allows
the designer the freedom of constructing the FSM without
having to worry about the number of states.  The procedure
introduced below, can be invoked to clean up unnecessary states.

The minimization process works by identifying equivalence
classes of states and replacing all the states within
an equivalence class by one state.  Two states A and B are
in the same equivalence class if and only if properties
E1 and E2 are true.
\begin{itemize}
    \item E1: $\lambda (A,x) = \lambda (B,x)$ for all x
    \item E2: $\delta(A,x) = \delta(B,x)$ for all x
\end{itemize}

$\lambda(A,x)$ is the output of a FSM in state A with input x.
$\delta(A,x)$ is the next state of a FSM in state A with input x.
For example, consider the following state table:

$$
\begin{array} {c||c|c}
    CS \bs x & 0     & 1    \\ \hline \hline
    Q0       & Q0,0  & Q1,0 \\ \hline
    Q1       & Q0,0  & Q2,0 \\ \hline
    Q2       & Q0,0  & Q3,1 \\ \hline
    Q3       & Q0,0  & Q3,1 \\
\end{array} $$

Lets determine what $\lambda (Q2,0)$ is equal to.  Since $\lambda$
describes the output, we have to ask ourselves, what the output of
the FSM described by the state table when it is in state Q2 and
the input is 0?  Clearly, the answer is 0.  The following lists
some other values of $\lambda$.
\begin{itemize}
    \item $\lambda (Q2,1) =  1$
    \item $\lambda (Q2,0) =  0$
    \item $\lambda (Q3,1) =  1$
\end{itemize}
From this we can conclude that states Q2 and Q3 have the same output
for each possible input.  In other words:
$\lambda (Q2,x) = \lambda (Q3,x), \forall x$
The $\forall$ symbol is interpreted as "for all".  Now, lets look at
the $\delta$ function for the states Q2 and Q3.
\begin{itemize}
    \item $\delta(Q2,0) = Q0$
    \item $\delta(Q2,1) = Q3$
    \item $\delta(Q2,0) = Q0$
    \item $\delta(Q3,1) = Q3$
\end{itemize}
From this we we can conclude that $\delta (Q2,x) = \delta (Q3,x) \forall x$
Since, states Q2 and Q3 satisfy property E1 and E2 then Q2 and Q3 are in the
same equivalence class -- they are equal.  Hence, states Q2 and Q3 can be
replaced a single state called Qnew.  The minimal state is formed by
eliminating all but one state of the equivalence class from the state table
(knock out rows).  Then all occurrence of the equivalent states
are replaced by Qnew.  When this is done to our original state table
we get the minimal state table below.

$$
\begin{array} {c||c|c}
    CS \bs x & 0     & 1    \\ \hline \hline
    Q0       & Q0,0  & Q1,0 \\ \hline
    Q1       & Q0,0  & Qnew,0 \\ \hline
    Qnew     & Q0,0  & Qnew,1 \\
\end{array} $$

This method is a bit slow and would be difficult to perform on large
state table.  The following presents a procedure that is guarantee to
produce a minimal state table every time.

\subsection{State Minimization Procedure}
The state minimization procedure starts with the largest
possible equivalence classes (an optimistic view)
and then iteratively fractures the equivalence classes into
smaller classes.

\begin{description}
    \item[Step 1]  Put states into equivalence classes based on their
        output.  Give each equivalence class a name

    \item[Step 2]  Interpret the next state values of each state in
        terms of the equivalence classes.  These will be called the next
        class values of a state.

    \item[Step 3]  In each equivalence class identify and move any
        state which has different next class values.  Any subset of states
        which disagree with the other states in the class, but agree with
        each other should be moved together.  The moved state(s) should be
        put into their own new equivalence class.

    \item [Step 4] If states were moved in Step 3 then go back to Step 2.
        Otherwise the partitioning of the equivalence class is done and the
        minimal state table should be formed.  This is done by interpreting
        each equivalence class as a state.  The next class entries of
        all the states must agree (otherwise we would have partitioned in Step
        2) and are listed as the "next state" entries of the equivalence class.
        The outputs must be determined by looking up the output values in
        the original state table.
\end{description}

If all this sounds very cryptic, thats all right, we'll work through an
example to make each of the steps clear. Consider the following state
table:
$$
\begin{array} {c||c|c}
    CS \bs x & 0     & 1    \\ \hline \hline
    Q0       & Q4,1  & Q3,0 \\ \hline
    Q1       & Q3,1  & Q4,1 \\ \hline
    Q2       & Q0,1  & Q3,0 \\ \hline
    Q3       & Q1,1  & Q4,1 \\ \hline
    Q4       & Q3,1  & Q4,0 \\
\end{array} $$

\begin{description}
    \item [Step 1]  The outputs are used to partition
        the states into equivalence classes.  Since there two different
        output combinations exhibited in the state table (1,0) and (1,1)
        our initial partition consists of two equivalence classes.
        This is the most optimistic minimization that can be obtained;
        we are only using rule E1 at this point.

        \begin{tabular}{|c||c|c|c||c|c|}\hline
            Class & \multicolumn{3}{c||}A & \multicolumn{2}{c|} B \\ \hline
            State & Q0 & Q2 & Q4 & Q1 & Q3 \\ \hline
        \end{tabular}

    \item[Step 2] The next state values for each of the states
        Q0 $\ldots$ Q4 are interpreted in terms of the class partitions A,B.
        For example the next state entries for Q0 are Q4,Q3 for x equal to
        0,1 respectively.  State Q4
        is in class A and state Q3 is in state B.  Hence, the next class
        entries for state Q0 are A,B.  This pair of next class entries is
        written below the state Q0.  This procedure is carried out for the
        remaining states Q1 $\ldots$ Q4 keeping the ordering x equal 0,1 consistent.
        The following table results.

        \begin{tabular}{|c||c|c|c||c|c|}\hline
            Class      & \multicolumn{3}{|c||} A & \multicolumn{2}{c|} B \\ \hline
            State      & Q0 & Q2 & Q4 & Q1 & Q3 \\ \hline
            Next Class & AB & AB & BA & BA & BA \\ \hline
        \end{tabular}

    \item[Step 3]
        It is clear that state Q4 does not belong in class A; it
        does not have the same next class as Q0,Q4.
        State Q4 is removed from class A and put into its own class,
        called C.

        Since Q4 has the same next class values as the states in class B
        then can state Q4 be moved into equivalence class B?  No, state
        Q4 has different outputs from the states in equivalence class B,
        that is why it was put into a different class initially.  In this phase
        (steps 2,3) of the state minimization process the equivalence class are
        only made smaller, class are \textit{ never} merged.

    \item [Step 4]
        Since states were moved in step 3 we must go back to step 2 with
        the classes A,B,C.

    \item [Step 2]  The next class entries are written down
        for each state.  If you were lazy you might skip writing
        down the next class entries for state Q4 because it cannot
        be different from the other members of its equivalence class; it
        is by itself.

        \begin{tabular}{|c||c|c||c|c||c|}\hline
            Class      & \multicolumn{2}{|c||} A & \multicolumn{2}{c||} B & C\\ \hline
            State      & Q0 & Q2 & Q1 & Q3 & Q4 \\ \hline
            Next Class & CB & AB & BC & BC & BC \\ \hline
        \end{tabular}

    \item [Step 3]
        We see that states in class A have different next class entries so
        one of them must be moved into a new class.  The choice of which
        state to move is arbitrary, so we will move Q2 into class D.
        Since states were moved in step 3 we must go back to step 2 with
        the classes A,B,C,D.

    \item[Step 2]
        \begin{tabular}{|c||c||c|c||c||c|}\hline
            Class      & A & \multicolumn{2}{c||} B & C & D\\ \hline
            State      & Q0 & Q1 & Q3 & Q4 & Q2  \\ \hline
            Next Class & CB & BC & BC & BC & AB  \\ \hline
        \end{tabular}

    \item[Step 3]
        All the states within the classes are the same so no
        partitioning is required.

    \item[Step 4]
        There was no partitioning in step 3 so we can now form the minimal
        state table.  This is done by listing each class on a row of a
        state table.  The next class entries are listed as the next state
        entries in the state table.  The outputs are determined by looking
        at the outputs for one of the states within the equivalence class.

        $$
        \begin{array} {c||c|c}
            CS \bs x & 0     & 1    \\ \hline \hline
            A       & C,1  & B,0 \\ \hline
            B       & B,1  & C,1 \\ \hline
            C       & B,1  & C,0 \\ \hline
            D       & A,1  & B,0 \\
        \end{array} $$

        Realize that it does not matter what the states are
        called, so at this point you could revert back to some state notation
        that you need/want for the design process.
\end{description}

The state minimization problem can be made to handle arbitrarily
large state tables.  Increasing the number of input bits increases the
number of entries in the  Next Class row as is seen in the following example.

$$
\begin{array} {c||c|c|c|c}
    CS \bs x_1 x_0 & 00    & 01   & 11    & 10   \\ \hline \hline
    Q0       & Q0,1  & Q5,1 & Q1,1  & Q4,1 \\ \hline
    Q1       & Q2,0  & Q1,0 & Q3,0  & Q5,0 \\ \hline
    Q2       & Q4,1  & Q1,1 & Q5,1  & Q0,1 \\ \hline
    Q3       & Q3,1  & Q2,1 & Q4,1  & Q1,1 \\ \hline
    Q4       & Q5,0  & Q4,0 & Q0,0  & Q2,0 \\ \hline
    Q5       & Q1,1  & Q1,1 & Q2,1  & Q3,1 \\
\end{array} $$

\begin{description}
    \item[Step 1]  The outputs are used to set up an initial partition.

        \begin{tabular}{|c||c|c|c|c||c|c|}\hline
            Class      & \multicolumn{4}{|c||} A & \multicolumn{2}{c|} B\\ \hline
            State      & Q0   & Q2   & Q3   & Q5   & Q1   & Q4   \\ \hline
        \end{tabular}

    \item[Step 2]  The next class entries are recorded for each state.  Since
        there are four input combinations each state will have four next class
        entries.  The input order 00,01,11,10 will be maintained.

        \begin{tabular}{|c||c|c|c|c||c|c|}\hline
            Class      & \multicolumn{4}{|c||} A & \multicolumn{2}{c|} B \\ \hline
            State      & Q0   & Q2   & Q3   & Q5   & Q1   & Q4   \\ \hline
            Next Class & AABB & BBAA & AABB & BBAA & ABAA & ABAA \\ \hline
        \end{tabular}

    \item[Step 3]  Since the next class entries in class A do not agree,
        some states will have to be moved.  Since both states Q2 and
        Q5 agree on their next class entries they will be moved as a pair
        into an new class C.

    \item[Step 2]  The next class entries for the new partition of
        the states is determined.

        \begin{tabular}{|c||c|c||c|c||c|c|}\hline
            Class      & \multicolumn{2}{|c||} A & \multicolumn{2}{c||} B & \multicolumn{2}{c|} C\\ \hline
            State      & Q0   & Q3   & Q1   & Q4   & Q2   & Q5 \\ \hline
            Next Class & ACBB & ACBB & CBAC & CBAC & BBCA & BBCA \\ \hline
        \end{tabular}

    \item[Step 3]Since no states were moved, the state table can be written.

        $$
        \begin{array} {c||c|c|c|c}
            CS \bs x_1 x_0 & 00  & 01  & 11   & 10   \\ \hline \hline
            A        & A,1 & C,1 & B,1  & B,1 \\ \hline
            B        & C,0 & B,0 & A,0  & C,0 \\ \hline
            C        & B,1 & B,1 & C,1  & A,1 \\
        \end{array} $$
\end{description}

The process of determining a minimal state table is a rather simple procedure
which could easily be automated by a program which reads in a state table and
output a minimal state table.

\section{State Assignment}
State assignment is the process of assigning binary codes to the
symbolic states.   This step does not need to be performed when a
one's hot encoding is used because the binary
code for each state is predetermined.  This step is only performed
when the states of the FSM are to be encoded using a dense coding.

Before the assignment can proceed the number of flip flops required
must be quantified.  In general, if a FSM has $N$ states then it
will require $\lceil \log_2(N) \rceil$ flip flops.

The state assignment does influence the amount of combinational
logic required to realize the FSM.  Consider the following state
table, it will be realized using D flip flops.

\begin{tabular} {c||c|c}
    CS $\bs$ X & 0     & 1    \\ \hline \hline
    A       & C,0  & D,0 \\ \hline
    B       & A,1  & B,0 \\ \hline
    C       & C,0  & D,1 \\ \hline
    D       & A,1  & B,1 \\
\end{tabular}

\begin{tabular}{l|l}
    Assignment 1 & Assignment 2 \\ \hline
    \begin{tabular} {c||c}
        State   & Q1 Q0 \\ \hline
        A       & 00 \\ \hline
        B       & 01 \\ \hline
        C       & 10 \\ \hline
        D       & 11 \\
    \end{tabular}
    &
    \begin{tabular} {c||c}
        State   & Q1 Q0 \\ \hline
        A       & 00 \\ \hline
        B       & 01 \\ \hline
        C       & 11 \\ \hline
        D       & 10 \\
    \end{tabular}  \\ \hline

    \begin{tabular} {c||c|c}
        Q1 Q0 $\bs$ X & 0  & 1  \\ \hline \hline
        00      & 10  & 11 \\ \hline
        01      & 00  & 01 \\ \hline
        11      & 00  & 01 \\ \hline
        10      & 10  & 11 \\
    \end{tabular}
    &
    \begin{tabular} {c||c|c}
        Q1 Q0 $\bs$ X & 0  & 1  \\ \hline \hline
        00      & 11  & 10 \\ \hline
        01      & 00  & 01 \\ \hline
        11      & 11  & 10 \\ \hline
        10      & 00  & 01 \\
    \end{tabular}  \\ \hline

    \begin{tabular}{l}
        D1 = Q0' \\
        D0 = X \\
    \end{tabular}
    &
    \begin{tabular}{l}
        D1 = Q1'*Q0' + Q1*Q0 \\
        D0 = X'*Q1'*Q0' + X*Q1'*Q0 + X'*Q1*Q0 + X*Q1*Q0' \\
    \end{tabular} \\
\end{tabular}

Clearly, assignment 1 produces a more minimal realization.  Incredibly,
this drastic change in the realization is the result of
changing the assignments of states C and D.  Thus, we are motivated to
find a state assignment that minimizes the amount of combinational logic
required to realize a FSM.  Unfortunately, this problem is not solvable
in practice as the following discussion eludes to.

How many state assignments are possible in a FSM with $N$ states?
The answer is too many.  Let the states be labeled $Q_1 \ldots Q_N$.

\begin{itemize}
    \item State $Q_N$ may have $N$ possible binary representations,
    \item Since one combination was used by state $Q_N$ there are N-1 combinations
        left to choose from for state $Q_{N-1}$.
    \item This process continues, each state using one combination till
        we get to the final state.
    \item State $Q_1$ has 1 possible binary representation.
\end{itemize}

These choices are independent so in total there are $N*(N-1) * \ldots * 1 = N!$
different choices of state assignments.  This is a stunningly large number
for even a humble FSM with 8 states; 40,320 different state assignments.
Thus, we cannot check for the best assignment by enumerating all the state
assignment and looking for the best.  Coupled with
this difficulty is the fact that determining an \SOPmin expression
is a rather daunting problem in of itself.

Given the preceding discussion, it should come as no surprise
that determining a state assignment which minimizes the
amount of combinational logic is, in practice, not feasible.
Thus, if we want an assignment that minimizes the amount
of combinational logic then we must be guided by \textit{ heuristics}.
A heuristic is a rule of thumb.

The heuristics for the state assignment problems are based on assigning
pairs of states \textit{ adjacent} assignments.  Two states are adjacent if
their binary representation differ by a single bit.

The following pairs of states should be given adjacent state assignments.
Highest priority should be given to the first heuristic and the
lowest priority should be given to the last heuristic.

\begin{tabular}{ll}
    First  & States with the same next states for all inputs \\
    Second & States with the same next states for some inputs \\
    Third  & Next states \\
    Fourth & States with the same outputs for all inputs \\
    Fifth  & States with the same outputs for some inputs \\
\end{tabular}

These heuristics will be explained with the aide of an
example.  Consider the state table used in the example
above.

\begin{tabular} {c||c|c}
    CS $\bs$ X & 0     & 1    \\ \hline \hline
    A       & C,0  & D,0 \\ \hline
    B       & A,1  & B,0 \\ \hline
    C       & C,0  & D,1 \\ \hline
    D       & A,1  & B,1 \\
\end{tabular}

The first heuristic states that two state which have the same
next states for all inputs should be given adjacent binary
representation.  Looking at the state table we see that states
A and C have the same next states C,D for X equal 0,1
respectively.  So the pair of states (A,C) should be given
adjacent binary assignments.  Likewise, the pair of
states (B,D) have the same next states A,B for all inputs.
Thus, from this step the state pairs (A,C) and (B,D) should be
given adjacent binary representations.

The second heuristic state that states with the same next state
for a single input should be given adjacent assignments.  Typically,
state pairs are only written down once.  Thus, we will not include
(C,D) or (A,B).  There are no other pairs of states in the state
table with the same next state for a single input.

The third heuristic states that the next states of a given state
should be given adjacent assignments.  State A has next states
(C,D), so this pair of states will be given adjacent assignments.
Likewise, the pair of states (A,B) will be given adjacent binary
representations.

The fourth and fifth heuristics are analogous to the first and
second heuristic except that these two heuristics focus on the
output.  No two states have the same output for all inputs.

From the fifth heuristic we have that the state pair (A,C),
(B,D), (A,B) and (C,D) should be adjacent.  Since all of these
have already been included above, none of them need to be noted.

The heuristics are usually written down in a more concise form
\begin{enumerate}
    \item (A,C) (B,D)
    \item none
    \item (C,D) (A,B)
    \item none
    \item none
\end{enumerate}

In the example at the beginning of this section, assignment 1 satisfies
all the heuristics while assignment 2 only satisfies the third heuristic.
This example gives evidence that the heuristics provide good
state assignments.  It may not be possible to satisfy all the
assignments.  In such cases, try to satisfy as many as possible;
giving preference to the highest.  Remember that these are only
heuristics, there is no guarantee that they will generate the best
possible assignment.

When a state table has more than 4 states a kmap may be used to help
organize the states according to the heuristics.  For example,
consider the state table defined below.

$$
\begin{array} {c||c|c}
    State   & 0   & 1   \\ \hline
    A       & C,0  & B,0 \\ \hline
    B       & E,1  & D,0 \\ \hline
    C       & E,1  & D,0 \\ \hline
    D       & F,1  & G,1 \\ \hline
    E       & E,0  & G,0 \\ \hline
    F       & A,0  & A,0 \\ \hline
    G       & A,1  & A,1 \\
\end{array} $$

The heuristics are applied to this state table generating the following
list of adjacencies.

\begin{enumerate}
    \item (B,C) (F,G)
    \item (B,E) (C,E) (D,E)
    \item (C,B) (E,D) (F,G) (E,G)
    \item (A,E,F) (B,C) (D,G)
    \item too many
\end{enumerate}

Now we play a game where you try to satisfy as many of the
heuristics as possible -- highest priority first.  A kmap will
aide in positioning the states in order to satisfy as many
heuristics as possible.  The feature of a kmap that makes it useful
in this situation is that adjacent cells of a kmap differ by a single
bit.  Hence, placing two states next to each other in a kmap will
give them binary representations which differ by a single bit.
It would be a good time to give the reset state the binary
code $0 \ldots 0$ by placing it in the 0 corner of the kmap.
In the above example, state A will be taken to be the reset states.

$$
\begin{array} {c||c|c|c|c}
    Q_2 \bs Q_1 Q_0 & 00 & 01 & 11 & 10 \\ \hline \hline
    0           &  A & C  & B  &    \\ \hline
    1           &  D & E  & G  & F  \\
\end{array} $$

The cells of this kmap are filled with the states in order to satisfy
the constraints determined earlier.  The results of this kmap can be
used to determine the state assignment table.

\begin{tabular} {c||c|c}
    State   & $Q_2 Q_1 Q_0$
    A       & 000 \\ \hline
    B       & 011 \\ \hline
    C       & 001 \\ \hline
    D       & 100 \\ \hline
    E       & 101 \\ \hline
    F       & 110 \\ \hline
    G       & 111 \\
\end{tabular}

\section{Exercises}
\begin{enumerate}

    \item Determine the minimal state table for the following
        state table.

        \begin{tabular}{c||c|c}
            CS $\bs$ X &  0     &  1 \\ \hline \hline
            s1  &  s2,0  &  s3,0 \\ \hline
            s2  &  s4,0  &  s5,0 \\ \hline
            s3  &  s6,0  &  s7,0 \\ \hline
            s4  &  s8,0  &  s9,0 \\ \hline
            s5  &  s10,0 &  s11,0 \\ \hline
            s6  &  s12,0 &  s13,0 \\ \hline
            s7  &  s14,0 &  s15,0 \\ \hline
            s8  &  s1,0  &  s3,0 \\ \hline
            s9  &  s1,0  &  s3,0 \\ \hline
            s10 &  s1,0  &  s3,0 \\ \hline
            s11 &  s1,0  &  s2,0 \\ \hline
            s12 &  s2,1  &  s1,0 \\ \hline
            s13 &  s2,0  &  s1,0 \\ \hline
            s14 &  s2,0  &  s1,0 \\ \hline
            s15 &  s2,0  &  s1,0 \\
        \end{tabular}

    \item Determine the MIE's for one's hot implementation of the state
        diagram in Figure~\ref{fig:MinState}.  Make sure to minimize the
        number of states before determining the MIEs.
        \begin{figure}[ht]
            %\centerline{\includegraphics{figure=../chapter09/Fig/MinState}}
            \caption{An Mealy FSM state diagram.}
            \label{fig:MinState}
        \end{figure}

    \item Determine a good state assignment for the following FSM assuming
        that state A is assigned the binary code $(Q_3 Q_2 Q_1 Q_0) = 0000$.

        \begin{tabular}{c||c|c}
            CS $\bs$ X&  0    &  1 \\ \hline \hline
            A   &  B,0  &  G,0 \\ \hline
            B   &  E,1  &  H,1 \\ \hline
            C   &  C,0  &  D,1 \\ \hline
            D   &  B,0  &  A,0 \\ \hline
            E   &  B,1  &  D,0 \\ \hline
            F   &  F,0  &  E,1 \\ \hline
            G   &  D,0  &  A,1 \\ \hline
            H   &  G,1  &  D,0 \\
        \end{tabular}

    \item Determine the minimal state table for the following state table

        \begin{tabular}{c||c|c|c|c}
            CS $\bs$ X1,X0& 00    & 01    & 11    & 10    \\ \hline
            A    & A,0    & A,0    & A,0    & A,0    \\ \hline
            B    & A,0    & A,0    & A,0    & A,0    \\ \hline
            C    & A,0    & A,0    & A,0    & A,0    \\ \hline
            D    & A,0    & A,0    & A,0    & A,0    \\ \hline
            E    & A,0    & A,0    & A,0    & A,0    \\ \hline
            F    & A,0    & A,0    & A,0    & A,0    \\ \hline
            G    & A,0    & A,0    & A,0    & A,0    \\ \hline
            H    & A,0    & A,0    & A,0    & A,0    \\
        \end{tabular}

    \item Determine a good state assignment for the following minimal state
        table.

        \begin{tabular}{c||c|c|c|c}
            CS $\bs$ X1,X0& 00    & 01    & 11    & 10    \\ \hline
            A    & A,0    & A,0    & A,0    & A,0    \\ \hline
            B    & A,0    & A,0    & A,0    & A,0    \\ \hline
            C    & A,0    & A,0    & A,0    & A,0    \\ \hline
            D    & A,0    & A,0    & A,0    & A,0    \\ \hline
            E    & A,0    & A,0    & A,0    & A,0    \\ \hline
            F    & A,0    & A,0    & A,0    & A,0    \\ \hline
            G    & A,0    & A,0    & A,0    & A,0    \\ \hline
            H    & A,0    & A,0    & A,0    & A,0    \\
        \end{tabular}

\end{enumerate}
