\section{Digital Design Body Of Knowledge}
The focus of this text is very much on the datapath and control approach.  To achieve this
end in a single semester, sacrafices in coverage must be made.  

\dirtree{%
	.1 Digital Design.
	.2 Numbering Systems.
	.3 Positional Numbering Systems.
	.4 Base 10 - Decimal.
	.4 Base 2 - Binary.
	.4 Base 16 - Hexadecimal.
	.3 Conversion Between Bases.
	.3 Word Size.
	.3 2's Complement.	
	.2 Representation of Logical Function.
	.3 Elementary Logical Functions.
	.3 Word Statement.
	.3 Truth Table.
	.3 Symbolic.
	.3 Circuit Diagram.
	.3 Hardware Description Languages.
	.3 Conversion Between Representations.
	.3 Timing Diagrams.
	.2 Logic Minimization.
	.3 Karnaugh Maps (Kmaps).
	.3 Kmaps for circuits with multiple outputs.
	.3 Kmaps to find POSmin.
	.3 Logic Minimization Software.
	.2 Combination Logic Building Blocks.
	.3 Decoder.
	.3 Multiplexers.
	.3 Adders.
	.3 Comparators.
	.3 Three-State Buffers.
	.3 Wire Logic.
	.3 Combination.
	.4 Arithmetic Statements.
	.4 Conditional Statements.
	.2 Primitive Sequential Circuits.
	.3 Basic Memory Element.
}
\vspace{0.5cm}
