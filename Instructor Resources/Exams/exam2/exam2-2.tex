\documentclass{article}
\usepackage{epsfig, latexsym}

\begin{document}

\newcommand{\SOPmin}{${\rm SOP}_{\rm min} \ $}
\newcommand{\POSmin}{${\rm POS}_{\rm min} \ $}
\newcommand{\bs}{\backslash}


\title{
\Huge{CSE 271 -- Spring 2001}\\
\normalsize{Exam 2}\\
\makebox[4in][l]{Name:}
SSN:}
\date{}

\maketitle{}

\begin{tabular}{llll}
\begin{tabular}{c||c}
D & Q+   \\ \hline
0 & 0 \\ \hline
1 & 1 \\
\end{tabular}
&
\begin{tabular}{c||c}
T & Q+   \\ \hline
0 & Q \\ \hline
1 & Q' \\
\end{tabular}
&
\begin{tabular}{c|c||c}
S & R & Q+   \\ \hline
0 & 0 & Q \\ \hline
0 & 1 & 0 \\ \hline
1 & 0 & 1 \\ \hline
1 & 1 & x \\
\end{tabular}
&
\begin{tabular}{c|c||c}
J & K & Q+   \\ \hline
0 & 0 & Q \\ \hline
0 & 1 & 0 \\ \hline
1 & 0 & 1 \\ \hline
1 & 1 & Q' \\
\end{tabular}
\\
\end{tabular}


\begin{enumerate}
\item {\bf (1 pt.)} How many 3:8 decoders are required to construct a 6:64 decoder?
\begin{description}
\item{a) }2
\item{b) }4
\item{c) }5
\item{d) }8
\item{e) }9
\end{description}

\item {\bf (1 pt.)}How many inputs do the AND gates inside a 2:4 decoder have?
\begin{description}
\item{a) }2
\item{b) }3
\item{c) }4
\item{d) }5
\item{e) }None of the above.
\end{description}

\item {\bf (1 pt.)}How many inputs does the OR gates inside a 16:1 mux have?
\begin{description}
\item{a) }2
\item{b) }3
\item{c) }4
\item{d) }16
\item{e) }Trick question, a mux does not have an OR gate.
\end{description}

\pagebreak{}
\item {\bf (1 pt.)} If a 4:16 decoder is constructed from 1:2 decoders, how
many of the 1:2 decoders get the select line $S_3$?
\begin{description}
\item{a) } 1
\item{b) } 2
\item{c) } 4
\item{d) } 8
\item{e) } None of the above.
\end{description}

\item {\bf (1 pt.)} Assuming a word size of 5 bits, interpret 11010 as a 2's complement
number.
\begin{description}
\item{a) }-10
\item{b) }-6 
\item{c) }-5 
\item{d) }-4 
\item{e) }None of the above.
\end{description}


\item {\bf (1 pt.)} Assuming a word size of 4 bits, determine the 2's complement
representation of -5.
\begin{description}
\item{a) }1101
\item{b) }1010
\item{c) }1011 
\item{d) }1001 
\item{e) }Cannot be done in 5 bits.
\end{description}

\item {\bf (1 pt.)} Assuming a word size of 4 bits and a 2's complement representation,
compute  1101 - 0011
\begin{description}
\item{a) } 0110
\item{b) } 1010
\item{c) } 1000
\item{d) } Invalid answer because overflow occurs.
\item{e) } None of the above.
\end{description}

\pagebreak
\item {\bf (1 pt.)} Which of $G,L,E$ below must be connected to the 
sel input of the mux so that:
\verb+if (X > Y) then Z = X else Z = Y;+

\psfig{figure=./Fig2/compare.eps,height=1in.}

\begin{tabular}{lll}
a) G & b)  L & c)  E \\ 
\end{tabular}

\item {\bf (1 pt.)} What is the identity of {\bf BOX} in the figure below so that: \\
\verb+if (control=0) then DATA = A else DATA = B;+

\psfig{figure=./Fig2/tsb.eps,height=1in.}

\begin{description}
\item{a) } TriState buffer
\item{b) } Mux
\item{c) } Decoder
\item{d) } Adder/Subtractor
\item{e) } Comparator
\end{description}

\item {\bf (1 pt.)} How many address lines does a 2Mx32 RAM have?
\begin{description}
\item{a) } 5
\item{b) } 11
\item{c) } 21
\item{d) } 31
\item{e) } 32
\end{description}

\item  {\bf (1 pt.)} A FSM has N states.  How many flip flops
does the resulting circuit have?
\begin{description}
\item{a) } 2
\item{b) } $log_2(N)$
\item{c) } N/2
\item{d) } N
\item{e) } $2^N$
\end{description}

\item {\bf (1 pt.)} Given the following state table and the state assignment,
determine the entry in the transition kmap marked with a *.
{\small
$$\begin{array}{lll}
$$\begin{array}{c||c|c}
        cs \bs X & 0   &  1   \\ \hline \hline
        A        & A,1 & C,1 \\ \hline
        B        & B,0 & A,0 \\ \hline
        C        & D,0 & A,0 \\ \hline
        D        & C,1 & B,1 \\ 
\end{array}$$
&
$$\begin{array}{c||c|c}
        state & Q_1 & Q_0    \\ \hline \hline
        A     & 1 & 0  \\ \hline
        B     & 0 & 0 \\ \hline
        C     & 1 & 1 \\ \hline
        D     & 0 & 1 \\ 
\end{array}$$
&
$$\begin{array}{c||c|c}
        Q_1 Q_0 \bs X & 0   &  1   \\ \hline \hline
        00       &     &     \\ \hline
        01       &     &     \\ \hline
        10       & *   &     \\ \hline
        11       &     &     \\ 
\end{array}$$\\
\end{array}$$}

\begin{description}
\item{a) }00,0
\item{b) }01,1
\item{c) }10,0
\item{d) }11,0
\item{e) }None of the above.
\end{description}

Given the transition kmap to the right, determine the MIEs
and OEs assuming that the FSM is implemented using a D flip flop.
{\small
$$\begin{array}{cll}
$$\begin{array}{c||c|c|c|c}
        Q \bs X_1 X_0 & 00  & 01  & 11  & 10  \\ \hline \hline
        0             & 1,1 & 1,1 & 0,1 & 0,1 \\ \hline
        1             & 0,0 & 1,0 & 1,0 & 0,0 \\ 
\end{array}$$
&
$$\begin{array} {c||c|c|c|c}
        Q \bs X_1 X_0 & 00 & 01 & 11 & 10 \\ \hline \hline
        0             &    &    &    &    \\ \hline
        1             &    &    &    &    \\ 
\end{array}$$
&
$$\begin{array} {c||c|c|c|c}
        Q \bs X_1 X_0 & 00 & 01 & 11 & 10 \\ \hline \hline
        0             &    &    &    &    \\ \hline
        1             &    &    &    &    \\ 
\end{array}$$ \\
Q^+,Z & D= & Z= \\
\end{array}$$}

\item {\bf (1 pt.)} What does $D=$ ?
\begin{description}
\item{a) }$Q'X_1'$ 
\item{b) }$Q'X_1' + QX_0$ 
\item{c) }$X_0'$
\item{d) }$X_1'$
\item{e) }$Q'$
\end{description}

\item {\bf (1 pt.)} What does $Z=$ ?
\begin{description}
\item{a) }$QX_0$ 
\item{b) }$QX_0 + Q'X_1'$ 
\item{c) }$X_0'$
\item{d) }$X_1$
\item{e) }$Q'$
\end{description}

\pagebreak
Questions 15-17 deal with the state diagram below:

\psfig{figure=./Fig2/sd.eps,height=1.5in.}

\item {\bf (1 pt.)} How many inputs and outputs does the FSM have?

\begin{description}
\item{a) }2 bits of input and 3 bits of output
\item{b) }2 bits of input and 2 bits of output
\item{c) }1 bit of input and 3 bits of output
\item{d) }1 bit of input and 2 bits of output
\item{e) }1 bit of input and 1 bit of output
\end{description}

\item {\bf (1 pt.)} Determine the entry in the state table marked with a *.
\marginpar{ \tiny $$ \begin{array} {c||c|c}
        cs \bs X & 0   &  1   \\ \hline \hline
        A        &     &     \\ \hline
        B        &     &     \\ \hline
        C        &     &  *  \\ \hline
        D        &     &     \\ 
\end{array} $$ }

\begin{description}
\item{a) }A,101
\item{b) }A,011
\item{c) }B,110
\item{d) }B,101
\item{e) }C,101
\end{description}

\item {\bf (1 pt.)} Which state(s), once entered, cannot be left (without
the aide of a reset).
\begin{description}
\item{a) }A and C
\item{b) }B
\item{c) }D
\item{d) }No such state(s) exists.
\item{e) }None of the above.
\end{description}

\pagebreak

Questions 18 and 19 deal with the following circuit and timing diagram:

\begin{figure}[ht]
\centerline{\psfig{figure=./Fig2/ExTim.eps,width=5in.,clip=}}
\end{figure}

\item {\bf (1 pts.)} What is the value of $Q_1$ at time 42?
\begin{description}
\item{a) }0
\item{b) }1
\item{c) }Changing rapidly.
\item{d) }Input changed too close to the edge to tell.
\item{e) }None of the above.
\end{description}

\item {\bf (1 pts.)} What is the value of $Q_2$ at time 58?
\begin{description}
\item{a) }0
\item{b) }1
\item{c) }Changing rapidly.
\item{d) }Input changed too close to the edge to tell.
\item{e) }None of the above.
\end{description}

\item {\bf (2 pts.)}Draw the state diagram for a FSM with 1 input and
1 output.  The output equals 1 when the FSM has seen the sequence 1010
Provide the semantics of the states.  The right-most bit is the oldest.

\end{enumerate}
\end{document}
