\documentclass{article}
\usepackage{epsfig, latexsym}

\begin{document}

\newcommand{\bs}{\backslash}


\title{
\Huge{CSE 271 -- Spring 2002}\\
\normalsize{Exam 2}\\
\makebox[4in][l]{Name:}
SSN:}
\date{}

\maketitle{}

\begin{enumerate}
\item {\bf (1 pt.)} How many inputs do the AND gates inside a 64:1 mux have?
\begin{description}
\item{a) }2
\item{b) }4
\item{c) }5
\item{d) }8
\item{e) }None of the above.
\end{description}

\item {\bf (1 pt.)}How many inputs do the AND gates inside a 4:16 decoder have?
\begin{description}
\item{a) }2
\item{b) }4
\item{c) }16
\item{d) }$2^{16}$
\item{e) }None of the above.
\end{description}

\item {\bf (1 pt.)}How many inputs does the OR gate inside a 4:16 decoder have?
\begin{description}
\item{a) }2
\item{b) }3
\item{c) }4
\item{d) }16
\item{e) }Trick question, a decoder does not have an OR gate.
\end{description}

\item {\bf (1 pt.)} Assuming a word size of 5 bits, interpret 10010 as a 2's complement
number.
\begin{description}
\item{a) }34
\item{b) }-2 
\item{c) }-12 
\item{d) }2
\item{e) }None of the above.
\end{description}

\pagebreak
\item {\bf (1 pt.)} Assuming a word size of 4 bits, determine the 2's complement
representation of -7.
\begin{description}
\item{a) }1101
\item{b) }1010
\item{c) }1011 
\item{d) }1001 
\item{e) }None of the above.
\end{description}

\item {\bf (1 pt.)} Which of $G,L,E$ below must be connected to the 
sel input of the mux so that \\
\verb+if (X < Y) then Z = X else Z = Y;+

\psfig{figure=./Fig2/compare.eps}

\begin{description}
\item{a) } G
\item{b) } L
\item{c) } E
\end{description}

\item {\bf (1 pt.)} For the cross coupled NAND gates shown below what 
is the value of $F^+$ when A=1 and B=1?
\begin{description}
\item{a) }0
\item{b) }1
\item{c) }F
\item{d) }F'
\item{e) }Changing rapidly.
\end{description}

\begin{figure}[ht]
\centerline{\psfig{figure=./Fig2/NANDs.eps,height=1.0in,clip=}}
\end{figure}

\pagebreak
For questions 8 and 9 use the following figure.
\begin{figure}[ht]
\centerline{\psfig{figure=./Fig2/ExTim2.eps,width=5in.,clip=}}
\end{figure}

\item {\bf (1 pt.)} What is the value of $Q_1$ at time 72?
\begin{description}
\item{a) }0
\item{b) }1
\item{c) }Changing rapidly.
\item{d) }Input changed too close to the edge to tell.
\item{e) }None of the above.
\end{description}

\item {\bf (1 pt.)} What is the value of $Q_2$ at time 37?
\begin{description}
\item{a) }0
\item{b) }1
\item{c) }Changing rapidly.
\item{d) }Input changed too close to the edge to tell.
\item{e) }None of the above.
\end{description}

\item {\bf (1 pt.)} How many 2:1 muxes are in an 8-bit register?
\begin{description}
\item{a) } 2
\item{b) } 3
\item{c) } 4
\item{d) } 8
\item{e) } 64
\end{description}

\pagebreak
\item {\bf (2 pt.)}A 4-bit (logical) shift register has the 
following truth table.  Determine the value on the $Q$ lines 
at time=85 assuming the sequence of inputs shown in the 
timing diagram. Note the D signal is the decimal representation
of the binary input.

\begin{tabular}{l|l|l||l}
clk		& $C_1 C_0$	& D & $Q^+$	\\ \hline
0,1,$\downarrow$& xx		& x & Q		\\ \hline
$\uparrow$ 	& 00		& x & Q		\\  \hline
$\uparrow$ 	& 01		& x & 0\&$Q>>1$	\\  \hline
$\uparrow$ 	& 10		& x & $Q<<1$\&0	\\  \hline
$\uparrow$ 	& 11		& D & D		\\
\end{tabular}

\psfig{figure=./Fig2/shift-time.eps,height=1.5in.}

\begin{description}
\item{a) } 0001
\item{b) } 0010
\item{c) } 0101
\item{d) } 1010
\item{e) } none of the above
\end{description}

\item {\bf (2 pt.)}A 4-bit up/down counter has the 
following truth table.  Determine the value on the $Q$ lines 
at time=85 assuming the sequence of inputs shown in the 
timing diagram for question 11. 

\begin{tabular}{l|l|l||l}
clk		& $C_1 C_0$	& D & $Q^+$	\\ \hline
0,1,$\downarrow$& xx		& x & Q		\\ \hline
$\uparrow$ 	& 00		& x & Q		\\  \hline
$\uparrow$ 	& 01		& x & Q+1 mod 16\\  \hline
$\uparrow$ 	& 10		& x & Q-1 mod 16\\  \hline
$\uparrow$ 	& 11		& D & D		\\
\end{tabular}

\begin{description}
\item{a) } 0101
\item{b) } 0110
\item{c) } 0100
\item{d) } 0011
\item{e) } none of the above
\end{description}

\pagebreak
\item {\bf (1 pt.)} How many address lines does a 256kx32 RAM have?
\begin{description}
\item{a) } 9
\item{b) } 16
\item{c) } 18
\item{d) } 19
\item{e) } 32
\end{description}

\item {\bf (1 pt.)} You want to write a value into a RAM.
Which of the following sequences is correct.
\begin{description}
\item{a) } assert data and address then assert CS and WE
\item{b) } assert CS adn WE then assert data and address
\item{c) } assert data and address then assert CS and RE
\item{d) } assert CS adn RE then assert data and address
\item{e) } Trick question, the order is unimportant.
\end{description}

Questions 11-13 deal with the state diagram below:

\psfig{figure=./Fig2/sd.eps, height=1.0in}

\item {\bf (1 pt.)} How many inputs and outputs does the FSM have?

\begin{description}
\item{a) }2 bits of input and 3 bits of output
\item{b) }2 bits of input and 2 bits of output
\item{c) }1 bit of input and 3 bits of output
\item{d) }1 bit of input and 2 bits of output
\item{e) }1 bit of input and 1 bit of output
\end{description}

\item {\bf (1 pt.)} Determine the entry in the state table marked with a *.
\marginpar{ \tiny $$ \begin{array} {c||c|c}
        cs \bs X & 0   &  1   \\ \hline \hline
        A        &     &     \\ \hline
        B        &     &     \\ \hline
        C        &     &  *  \\ \hline
        D        &     &     \\
\end{array} $$ }

\begin{description}
\item{a) }A,101
\item{b) }A,011
\item{c) }B,110
\item{d) }B,101
\item{e) }C,101
\end{description}

\pagebreak
\item {\bf (1 pt.)} Which state(s), once entered, cannot be left (without
the aide of a reset).
\begin{description}
\item{a) }A and C
\item{b) }B
\item{c) }D
\item{d) }No such state(s) exists.
\item{e) }None of the above.
\end{description}

\item {\bf (1 pt.)} Given the following state table and the state assignment,
determine the entry in the transition kmap marked with a *.
{\small
$$\begin{array}{lll}
$$\begin{array}{c||c|c}
        cs \bs X & 0   &  1   \\ \hline \hline
        A        & A,1 & C,1 \\ \hline
        B        & B,0 & A,0 \\ \hline
        C        & D,0 & A,0 \\ \hline
        D        & C,1 & B,1 \\
\end{array}$$
&
$$\begin{array}{c||c|c}
        state & Q_1 & Q_0    \\ \hline \hline
        A     & 1 & 0  \\ \hline
        B     & 0 & 0 \\ \hline
        C     & 1 & 1 \\ \hline
        D     & 0 & 1 \\
\end{array}$$
&
$$\begin{array}{c||c|c}
        Q_1 Q_0 \bs X & 0   &  1   \\ \hline \hline
        00       &     &     \\ \hline
        01       &     &     \\ \hline
        10       & *   &     \\ \hline
        11       &     &     \\
\end{array}$$\\
\end{array}$$}

\begin{description}
\item{a) }00,0
\item{b) }01,1
\item{c) }10,0
\item{d) }11,0
\item{e) }None of the above.
\end{description}



Given the transition kmap to the right, determine the MIEs
and OEs assuming that the FSM is implemented using a D flip flop.
{\small
$$\begin{array}{cll}
$$\begin{array}{c||c|c|c|c}
        Q \bs X_1 X_0 & 00  & 01  & 11  & 10  \\ \hline \hline
        0             & 1,1 & 1,1 & 0,1 & 0,1 \\ \hline
        1             & 0,0 & 1,0 & 1,0 & 0,0 \\
\end{array}$$
&
$$\begin{array} {c||c|c|c|c}
        Q \bs X_1 X_0 & 00 & 01 & 11 & 10 \\ \hline \hline
        0             &    &    &    &    \\ \hline
        1             &    &    &    &    \\
\end{array}$$
&
$$\begin{array} {c||c|c|c|c}
        Q \bs X_1 X_0 & 00 & 01 & 11 & 10 \\ \hline \hline
        0             &    &    &    &    \\ \hline
        1             &    &    &    &    \\
\end{array}$$ \\
Q^+,Z & D= & Z= \\
\end{array}$$}
\item {\bf (1 pt.)} What does $D=$ ?
\begin{description}
\item{a) }$Q'X_1'$
\item{b) }$Q'X_1' + QX_0$
\item{c) }$X_0'$
\item{d) }$X_1'$
\item{e) }$Q'$
\end{description}

\pagebreak
\item {\bf (1 pt.)} What does $Z=$ ?
\begin{description}
\item{a) }$QX_0$
\item{b) }$QX_0 + Q'X_1'$
\item{c) }$X_0'$
\item{d) }$X_1$
\item{e) }$Q'$
\end{description}


\item {\bf (4 pt.)} Show how to determine the smallest value in a 
128x8 RAM using the following code.  
\begin{verbatim}
    for(i=0; i<128; i++)
        if (M[i] < min) then min=M[i];
\end{verbatim}
You may assume that the \verb+i+ counter is initialized to 0 and
operate acording to the truth table shown in problem 12 (of course its 
a wider counter).  The\verb+min+ register is initialized to to 256 and
should load when its control input is 1, otherwise it holds.  The RAM
already contains data; you can hardwire the RAM control inputs.  You
can use any other building blocks you need.  Your solution will be scored 
as follows:
\begin{description}
\item 2 points for correct data flow
\item 1 point for correct control manipulation
\item 1 point for correct wire sizes
\end{description}

\end{enumerate}
\end{document}
