\documentclass{article}
\usepackage{epsfig, latexsym}

\begin{document}

\newcommand{\SOPmin}{${\rm SOP}_{\rm min} \ $}
\newcommand{\POSmin}{${\rm POS}_{\rm min} \ $}
\newcommand{\bs}{\backslash}


\title{
\Huge{CSE 271 -- Fall 2000}\\
\normalsize{Exam 2}\\
\makebox[4in][l]{Name:}
SSN:}
\date{}

\maketitle{}

\begin{tabular}{llll}
\begin{tabular}{c||c}
D & Q+   \\ \hline
0 & 0 \\ \hline
1 & 1 \\
\end{tabular}
&
\begin{tabular}{c||c}
T & Q+   \\ \hline
0 & Q \\ \hline
1 & Q' \\
\end{tabular}
&
\begin{tabular}{c|c||c}
S & R & Q+   \\ \hline
0 & 0 & Q \\ \hline
0 & 1 & 0 \\ \hline
1 & 0 & 1 \\ \hline
1 & 1 & x \\
\end{tabular}
&
\begin{tabular}{c|c||c}
J & K & Q+   \\ \hline
0 & 0 & Q \\ \hline
0 & 1 & 0 \\ \hline
1 & 0 & 1 \\ \hline
1 & 1 & Q' \\
\end{tabular}
\\
\end{tabular}


\begin{enumerate}
\item {\bf (1 pt.)} How many 2:4 decoders are required to construct a 6:64 decoder?
\begin{description}
\item{a) }3
\item{b) }16
\item{c) }21
\item{d) }32
\item{e) }63
\end{description}

\item {\bf (1 pt.)}How many inputs do the AND gates inside a 4:1 mux have?
\begin{description}
\item{a) }2
\item{b) }3
\item{c) }4
\item{d) }5
\item{e) }None of the above.
\end{description}

\item {\bf (1 pt.)} How many data outputs does a 4:1 mux have?
\begin{description}
\item{a) }1
\item{b) }2
\item{c) }4
\item{d) }6
\item{e) }None of the above.
\end{description}

\pagebreak{}
\item {\bf (1 pt.)} If a 16:1 mux is constructed from 4:1 muxes, how
many of the 4:1 muxes get the select line $S_3 S_2$?
\begin{description}
\item{a) } 1
\item{b) } 4
\item{c) } 5
\item{d) } 6
\item{e) } None of the above.
\end{description}

\item {\bf (1 pt.)} Your friend is building a 4 bit adder using
a truth table.  How many rows will the truth table have?
\begin{description}
\item{a) } 4
\item{b) } 8
\item{c) } 16
\item{d) } 64
\item{e) } 256
\end{description}

\item {\bf (1 pt.)} How many bits of output does a 4 bit adder have? Include overflow.
\begin{description}
\item{a) } 4
\item{b) } 5
\item{c) } 8
\item{d) } 9
\item{e) } 16
\end{description}

\item {\bf (1 pt.)} Assuming a word size of 4 bits, interpret 1010 as a 2's complement
number.
\begin{description}
\item{a) }-2
\item{b) }-5 
\item{c) }-6 
\item{d) }-10 
\item{e) }None of the above.
\end{description}

\pagebreak{}
\item {\bf (1 pt.)} Assuming a word size of 4 bits, determine the 2's complement
representation of -3.
\begin{description}
\item{a) }1100
\item{b) }1011
\item{c) }1100 
\item{d) }1101 
\item{e) }None of the above.
\end{description}

\item {\bf (1 pt.)} Assuming a word size of 4 bits and a 2's complement representation,
compute  1101 - 0011
\begin{description}
\item{a) } 0110
\item{b) } 1010
\item{c) } 1000
\item{d) } Invalid answer because overflow occurs.
\item{e) } None of the above.
\end{description}

\item {\bf (1 pt.)} What is $F^+$ for the following circuit when
$A=0$ and $B=0$?
\begin{description}
\item {a) } F 
\item {b) } F' 
\item {c) } 0 
\item {d) } 1 
\item {e) } None of the above. 
\end{description}

\psfig{figure=./Fig2/NANDs.eps} 

\pagebreak{}
\underline{For questions 11-14 use the circuit and timing diagram
show below}

\begin{figure}[ht]
\centerline{\psfig{figure=./Fig2/ExTim.eps,width=6in.,clip=}}
\end{figure}

\item {\bf (1 pt.)} What is the value of $Q_1$ at time=25nS?
\begin{description}
\item{a) }0 
\item{b) }1 
\item{c) }Toggling rapidly. 
\item{d) }Input(s) is (was) changed too close to the clock edge to tell.
\end{description}

\item {\bf (1 pt.)} What is the value of $Q_1$ at time=75nS?
\begin{description}
\item{a) }0 
\item{b) }1 
\item{c) }Toggling rapidly. 
\item{d) }Input(s) is (was) changed too close to the clock edge to tell.
\end{description}

\item {\bf (1 pt.)} What is the value of $Q_2$ at time=25nS?
\begin{description}
\item{a) }0 
\item{b) }1 
\item{c) }Toggling rapidly. 
\item{d) }Input(s) is (was) changed too close to the clock edge to tell.
\end{description}

\item {\bf (1 pt.)} What is the value of $Q_2$ at time=30nS?
\begin{description}
\item{a) }0 
\item{b) }1 
\item{c) }Toggling rapidly. 
\item{d) }Input(s) is (was) changed too close to the clock edge to tell.
\end{description}

\pagebreak{}
\item {\bf (1 pt.)}The counter shown in the figure below operates according to
truth table given.  Determine the count sequence on the
$Q$ lines.  Assume that $Q=0$ initially.

\begin{tabular}{l|l|l||l}
clk & $C_1 C_0$ & D & $Q^+$ \\ \hline
0,1,falling & xx & x & Q \\ \hline
rising & 00 & x & Q \\  \hline
rising & 01 & D & D \\  \hline
rising & 11 & x & Q+1 mod 8 \\ 
\end{tabular}
\begin{description}
\item{a) }0,6,0, $\ldots$ 
\item{b) }0,1,6,0, $\ldots$ 
\item{c) }0,1,6,7,0, $\ldots$ 
\item{d) }0,1,2,3,6,7,0 $\ldots$ 
\item{e) }0,1,2,3,4,5,6,6,6 $\ldots$ 
\end{description}
\psfig{figure=./Fig2/count.eps,height=1.5in.} 

\item {\bf (1 pt.)}Describe the output of the following circuit.
Hint, a TSB tristates when the control input equals 0.
\begin{description}
\item{a) }F(A,B,C)=A'B + AB'
\item{b) }F(A,B,C)=1
\item{c) }F(A,B,C)=A'BC' + AB'C
\item{d) }F(A,B,C)=A'BC + AB'C'
\item{e) }F(A,B,C)=Z
\end{description}
\psfig{figure=./Fig2/tri.eps,height=1.5in.} 

\pagebreak{}
\item Given the transition kmap to the right, determine the MIEs
assuming that the FSM is implemented using a JK flip flop.
{\small
$$\begin{array}{cll}
$$\begin{array}{c||c|c|c|c}
        Q \bs X_1 X_0 & 00  & 01  & 11  & 10  \\ \hline \hline
        0             & 1,1 & 1,1 & 0,1 & 0,1 \\ \hline
        1             & 0,0 & 1,0 & 1,0 & 0,0 \\ 
\end{array}$$
&
$$\begin{array} {c||c|c|c|c}
        Q \bs X_1 X_0 & 00 & 01 & 11 & 10 \\ \hline \hline
        0             &    &    &    &    \\ \hline
        1             &    &    &    &    \\ 
\end{array}$$
&
$$\begin{array} {c||c|c|c|c}
        Q \bs X_1 X_0 & 00 & 01 & 11 & 10 \\ \hline \hline
        0             &    &    &    &    \\ \hline
        1             &    &    &    &    \\ 
\end{array}$$ \\
Q^+,Z & J= & K= \\
\end{array}$$}

\item {\bf (1 pt.)} What does $J=$ ?
\begin{description}
\item{a) }$Q'X_1'$ 
\item{b) }$Q'X_1' + QX_0$ 
\item{c) }$X_0'$
\item{d) }$X_1'$
\item{d) }$Q'$
\end{description}

\item {\bf (1 pt.)} What does $K=$ ?
\begin{description}
\item{a) }$QX_0$ 
\item{b) }$QX_0 + Q'X_1'$ 
\item{c) }$X_0'$
\item{d) }$X_1$
\item{d) }$Q'$
\end{description}

\item {\bf (1 pt.)} Does the transition kmap you just worked on describe a 
Mealy or Moore FSM?
\begin{description}
\item{a) }This is a Mealy FSM.
\item{b) }This is a Moore FSM.
\end{description}

\end{enumerate}
\end{document}
