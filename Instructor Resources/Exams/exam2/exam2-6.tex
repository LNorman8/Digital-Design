\documentclass{article}
\usepackage{epsfig, latexsym}

\begin{document}

\newcommand{\bs}{\backslash}


\title{
\Huge{CSE 271 -- Spring 2003}\\
\normalsize{Exam 2}\\
\makebox[4in][l]{Name:}
SSN:}
\date{}

\maketitle{}

\begin{enumerate}
\item {\bf (1 pt.)}What is the equation for the $y_3$ output of a 4:16 decoder?
The select lines are denoted $s_i$ and the adta input is $d$.
\begin{description}
\item{a) }$s_2' s_1 s_0 d$
\item{b) }$s_3 s_2 s_1' s_0' d'$
\item{c) }$s_3 s_2 s_1' s_0' d$
\item{d) }$s_3' s_2' s_1 s_0 d$
\item{e) }None of the above.
\end{description}

\item {\bf (1 pt.)}How many 2:4 decoders does it take to build a 4:16 decoder?
\begin{description}
\item{a) }2
\item{b) }3
\item{c) }4
\item{d) }16
\item{e) }Trick question, you cannot build a 4:16 decoder with just 2;4 decoders.
\end{description}


\item {\bf (1 pt.)} If the delay through a single 2:1 mux is 1 unit of time then
what is the propagation delay through a 16:1 mux constructed from 2:1 muxes?
\begin{description}
\item{a) }2
\item{b) }4
\item{c) }8
\item{d) }15
\item{e) }None of the above.
\end{description}

\pagebreak
\item {\bf (1 pt.)} How many inputs do the AND gates in a 16:1 mux have?
\begin{description}
\item{a) }2
\item{b) }4
\item{c) }8
\item{d) }16
\item{e) }None of the above.
\end{description}

\item {\bf (1 pt.)} How many 2:1 muxes are needed to construct a 8x4x1 mux?
\begin{description}
\item{a) } 4
\item{b) } 8
\item{c) } 12
\item{d) } 18
\item{e) } 24
\end{description}

\item {\bf (1 pt.)} Which of the following is the equation for the carry
out of a full adder?
\begin{description}
\item{a) }cout = a  b + b  cin + a  cin
\item{b) }cout = a  b + b' cin + a  cin
\item{c) }cout = a  b + b  cin + a' cin
\item{d) }cout = a' b + b' cin + a  cin
\item{e) }cout = a' b + b  cin + a' cin
\end{description}

\item {\bf (1 pt.)} Assuming a word size of 5 bits, interpret 11010 as a 2's complement
number.
\begin{description}
\item{a) }-24
\item{b) }-2 
\item{c) }-12 
\item{d) }-6
\item{e) }None of the above.
\end{description}

\item {\bf (1 pt.)} Assuming a word size of 4 bits, determine the 2's complement
representation of -5.
\begin{description}
\item{a) }1011
\item{b) }1101
\item{c) }1011 
\item{d) }1001 
\item{e) }None of the above.
\end{description}

\pagebreak
\item {\bf (1 pt.)} Which staement is realized by the following circuit?

\psfig{figure=./Fig2/compare.eps}

\begin{description}
\item{a) } \verb+if (X > Y) then Z = X else Z = Y;+
\item{b) } \verb+if (X < Y) then Z = X else Z = Y;+
\item{c) } \verb+if (X >= Y) then Z = X else Z = Y;+
\item{d) } \verb+if (X <= Y) then Z = X else Z = Y;+
\item{e) } none of the above.
\end{description}

\item {\bf (1 pt.)} Which staement is realized by the following circuit?

\psfig{figure=./Fig2/combo.eps}

\begin{description}
\item{a) } \verb!if (X > Y) then Z = Y+7 else Z = Y+3;!
\item{b) } \verb!if (X < Y) then Z = Y+7 else Z = Y+3;!
\item{c) } \verb!if (X >= Y) then Z = Y+7 else Z = Y+3;!
\item{d) } \verb!if (X <= Y) then Z = Y+7 else Z = Y+3;!
\item{e) } none of the above.
\end{description}

\pagebreak
For questions 11-14 use the following figure

\psfig{figure=./Fig2/ExTim.eps,width=4in.}

\item {\bf (1 pt.)} What is the value of Q1 at time 25

\begin{tabular}{p{0.75in}p{0.75in}p{1.75in}}
a) 0 & b) 1 & c) toggling \\
\end{tabular}


\item {\bf (1 pt.)} What is the value of Q1 at time 75

\begin{tabular}{p{0.75in}p{0.75in}p{1.75in}}
a) 0 & b) 1 & c) toggling \\
\end{tabular}

\item {\bf (1 pt.)} What is the value of Q2 at time 25

\begin{tabular}{p{0.75in}p{0.75in}p{1.75in}}
a) 0 & b) 1 & c) toggling \\
\end{tabular}

\item {\bf (1 pt.)} What is the value of Q2 at time 75

\begin{tabular}{p{0.75in}p{0.75in}p{1.75in}}
a) 0 & b) 1 & c) toggling \\
\end{tabular}

\item {\bf (1 pt.)} Setup time is ...
\begin{description}
\item{a) } before the clock edge.
\item{b) } after the clock edge.
\item{c) } during the clock edge.
\end{description}


\pagebreak
\item {\bf (2 pt.)}A 4-bit (arithmetic) shift register has the 
following truth table.  Determine the value on the $Q$ lines 
at time=85 assuming the sequence of inputs shown in the 
timing diagram. Note the D signal is the decimal representation
of the binary input.

\begin{tabular}{l|l|l||l}
clk		& $C_1 C_0$	& D & $Q^+$	\\ \hline
0,1,$\downarrow$& xx		& x & Q		\\ \hline
$\uparrow$ 	& 00		& x & Q		\\  \hline
$\uparrow$ 	& 01		& x & $Q_3$\&$Q>>1$	\\  \hline
$\uparrow$ 	& 10		& x & $Q<<1$\& 0	\\  \hline
$\uparrow$ 	& 11		& D & D		\\
\end{tabular}

\psfig{figure=./Fig2/shift-time.eps,height=1.5in.}

\begin{description}
\item{a) } 0001
\item{b) } 0010
\item{c) } 0101
\item{d) } 1010
\item{e) } none of the above
\end{description}

\item {\bf (2 pt.)}A 4-bit up/down counter has the 
following truth table.  Determine the value on the $Q$ lines 
at time=85 assuming the sequence of inputs shown in the 
timing diagram for question 13. 

\begin{tabular}{l|l|l||l}
clk		& $C_1 C_0$	& D & $Q^+$	\\ \hline
0,1,$\downarrow$& xx		& x & Q		\\ \hline
$\uparrow$ 	& 00		& x & Q		\\  \hline
$\uparrow$ 	& 01		& x & Q+1 mod 16\\  \hline
$\uparrow$ 	& 10		& x & Q-1 mod 16\\  \hline
$\uparrow$ 	& 11		& D & D		\\
\end{tabular}

\begin{description}
\item{a) } 0101
\item{b) } 0110
\item{c) } 0100
\item{d) } 0011
\item{e) } none of the above
\end{description}

\pagebreak
\item {\bf (1 pt.)} How many 2:1 muxes are in an 8-bit register?
\begin{description}
\item{a) } 2
\item{b) } 3
\item{c) } 4
\item{d) } 8
\item{e) } 64
\end{description}

\item {\bf (1 pt.)} How many address lines does a 256kx32 RAM have?
\begin{description}
\item{a) } 9
\item{b) } 16
\item{c) } 18
\item{d) } 19
\item{e) } 32
\end{description}


\psfig{figure=./Fig2/square.eps}

\item {\bf (5 pt.)}In the circuit above assume that the 
control input to the registers is hard wired to "load".
In addition assume a regular clock and an inactive reset
line.  Assume that $S$ is initialized to 1 and $D$ is 
initialized to 3.  List the first four $S$ output.  
Briefly describe (in five words or less) the pattern of 
the $S$ outputs .

\end{enumerate}
\end{document}
