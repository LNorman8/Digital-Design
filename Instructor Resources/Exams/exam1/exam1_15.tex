\documentclass{article}
\usepackage{epsfig, latexsym}

\begin{document}

\newcommand{\SOPmin}{${\rm SOP}_{\rm min} \ $}
\newcommand{\POSmin}{${\rm POS}_{\rm min} \ $}
\newcommand{\bs}{\backslash}


\title{
\Huge{CMPEN 271 -- Spring 2012}\\
\normalsize{Return this exam!  No calculators!}\\
\normalsize{Exam 1}\\
\makebox[4in][l]{Name:} }
\date{}

\maketitle{}


\begin{enumerate}

\item {\bf (2 pts.)} Convert $100100_2$ to decimal.

\begin{tabular}{p{0.7in} p{0.7in} p{0.7in} p{0.7in} l}
a)20 & b)24  & c)40  & d)42  & e) none of the above
\end{tabular}

\item {\bf (2 pts.)} Convert $36_{10}$ to binary.

\begin{tabular}{p{0.7in} p{0.7in} p{0.7in} p{0.7in} l}
a) $010010_2$ & b) $100010_2$ & c) $100110_2$ & d) $100100_2$ & e) none of the above
\end{tabular}

\item {\bf (2 pts.)} Convert $36_{16}$ to binary.

\begin{tabular}{p{0.7in} p{0.7in} p{0.7in} p{0.7in} l}
a) $110010_2$ & b) $110100_2$ & c) $110110_2$ & d) $111000_2$ & e) none of the above
\end{tabular}

\item {\bf (2 pts.)} How many bits are required to represent the numbers
between 0 and $78_{10}$?

\begin{tabular}{p{0.7in} p{0.7in} p{0.7in} p{0.7in} l}
a) 6 & b) 7 & c) 78 & d) $2^{78}$ & e) none of the above
\end{tabular}

\item {\bf (1 pts.)} When representated as 4-bit binary numbers does 12 + 4 
generate overflow?

\begin{tabular}{p{0.7in} l}
a) yes & b) no  c) Trick question, 12 cannot be represented in 4-bit
\end{tabular}

\item {\bf (1 pt.)} How many 1's does the output column in a truth table for a 
5-input NAND gate have?

\begin{tabular}{p{0.7in} p{0.7in} p{0.7in} p{0.7in} l}
a) 0 & b) 1 & c) 5 & d) $2^{5}-1$ & e) $2^5$
\end{tabular}

\item {\bf (2 pt.)} Which expression is equivalent to (A'+B)'(B'+AC)?
\begin{description}
\item{a) } 0
\item{b) } 1
\item{c) } AB'
\item{d) } A'B + A'BC'
\item{e) } AB' + A'BC'
\end{description}

\pagebreak

\underline{For questions 8-11 assume F(A,B,C)= (A+B')C' + AB'C}

\item {\bf (2 pts.)} What does F(1,1,0) equal?

\begin{tabular}{p{0.7in} p{0.7in} p{0.7in} p{0.7in} l}
a) 0 & b) 1 & c) C & d) C' & e) none of these
\end{tabular}

\item {\bf (2 pts.)} What does F(1,0,C) equal?

\begin{tabular}{p{0.7in} p{0.7in} p{0.7in} p{0.7in} l}
a) 0 & b) 1 & c) C & d) C' & e) none of these
\end{tabular}

\item {\bf (1 pt.)} How many AND gates does it take to realize F
as is (do not simplify)?

\begin{tabular}{p{0.7in} p{0.7in} p{0.7in} p{0.7in} l}
a) 1 & b) 2 & c) 3 & d) 4 & e) none of these
\end{tabular}

\item {\bf (1 pt.)} How many OR gates does it take to realize F
as is (do not simplify)?

\begin{tabular}{p{0.7in} p{0.7in} p{0.7in} p{0.7in} l}
a) 1 & b) 2 & c) 4 & d) 5 & e) none of these
\end{tabular}

\underline{Utilize the following truth table for problems 12,13.}

\begin{tabular}{c|c|c||c|c}
A & B & C & F & G  \\ \hline \hline
0 & 0 & 0 & 1 & 1  \\ \hline
0 & 0 & 1 & 0 & 0  \\ \hline
0 & 1 & 0 & 0 & 0  \\ \hline
0 & 1 & 1 & 1 & 0  \\ \hline
1 & 0 & 0 & 1 & 1  \\ \hline
1 & 0 & 1 & 0 & 1  \\ \hline
1 & 1 & 0 & 1 & 0  \\ \hline
1 & 1 & 1 & 1 & 0  \\
\end{tabular} 

\item {\bf (1 pt.)} What function is described by $\prod M(0,4,5)$?

\begin{tabular}{p{0.7in} p{0.7in} p{0.7in} p{0.7in} l}
a) F & b) F' & c) G & d) G' & e) none of the above
\end{tabular}


\item {\bf (1 pt.)} How many product terms does the canonical SOP expression 
for F have?

\begin{tabular}{p{0.7in} p{0.7in} p{0.7in} p{0.7in} l}
a) 1 & b) 2 & c) 3 & d) 4 & e) 5
\end{tabular}

\underline{Utilize the following word statement for problems 14-15.}

Design a 4-input $a_1a_0b_1b_0$, 4-output 
$O_3O_2O_1O_0$ digital system.  $A=a_1a_0$ and $B=b_1b_0$ represent 
2-bit binary numbers.  The output should be the product (multiplication) 
of the inputs plus 5, that is $O=A*B+5$. 

\item {\bf (1 pt.)}How many rows will have the output $1011_2$?

\begin{tabular}{p{0.7in} p{0.7in} p{0.7in} p{0.7in} l}
a) 0 & b) 1 & c) 2 & d) 3 & e) None of the above.
\end{tabular}

\item {\bf (1 pt.)}How many rows of the truth table will have $O_0 = 1$?

\begin{tabular}{p{0.7in} p{0.7in} p{0.7in} p{0.7in} l}
a) 1 & b) 3 & c) 9 & d) 12 & e) None of the above.
\end{tabular}

\pagebreak

\underline{Utilize the following circuit diagram for problems 16,17.}

\begin{figure}[ht]
\rightline{\psfig{figure=./Fig1/cir_5.eps,width=4in,clip=}}
\end{figure}

\item {\bf (3 pts.)} What is the symbolic representation of 
$G(A,B,C)$ as shown?  \begin{description}
\item{a) } B'
\item{b) } BC + (B+C)C + B'
\item{c) } (BC)(A'C')C + B'
\item{d) } (BC)(A+C)'C + B'
\item{e) } None of the above.
\end{description}

\item {\bf (1 pts.)} What does G(0,1,1) equal?

\begin{tabular}{p{0.7in} p{0.7in} l}
a) 0 & b) 1 & c) None of the above
\end{tabular}

\item {\bf (3 pt.)} Determine the \SOPmin expression for \\
F(A,B,C,D)=$\Sigma$m(0,6,8,10,13,14,15)
\marginpar{ \small $$ \begin{array} {c||c|c|c|c}
        AB \bs CD & 00 & 01 & 11 & 10 \\ \hline \hline
        00        &    &    &    &    \\ \hline
        01        &    &    &    &    \\ \hline
        11        &    &    &    &    \\ \hline
        10        &    &    &    &    \\
\end{array} $$ }

\begin{description}
\item{a) } BCD' + ACD' + ABC' + A'B'C'D' + AB'C'D'
\item{b) } BCD' + ACD' + ABC' + B'C'D'
\item{c) } B'C'D' + AB'D' + BCD' + ABD
\item{d) } CD' + AB + B'C'D' 
\item{e) } None of the above.
\end{description}

\item {\bf (3 pt.)} Determine the \SOPmin expression for \\
F(A,B,C,D)=$\Sigma$m(3,6,9,12) + $\Sigma$d(0,4,7,8,14)
\marginpar{\small $$ \begin{array} {c||c|c|c|c}
        AB \bs CD & 00 & 01 & 11 & 10 \\ \hline \hline
        00        &    &    &    &    \\ \hline
        01        &    &    &    &    \\ \hline
        11        &    &    &    &    \\ \hline
        10        &    &    &    &    \\
\end{array} $$ }


\begin{description}
\item{a) } C'D' + AB' A'CD + BCD'
\item{b) } A'C' + A'B'D' + BC'D + AB'C
\item{c) } BD' + A'CD + AB'C'
\item{d) } BC'D' + AB'C' + A'CD + BCD'
\item{e) } None of the above.
\end{description}

\item {\bf (4 pt.)} Determine the \POSmin expression for \\
F(A,B,C,D)= (A+B'+D)(B+C')(B'+C'+D)

\begin{description}
\item{a) } (B+C')(A+B'+D')(C'+D)
\item{b) } (B+C'+D')(C'+D)(A+B'+D)
\item{c) } (A+B'+D)(B+C')(B'+C'+D)
\item{d) } (B+C)(A'+C)(B'+D)
\item{e) } None of the above.
\end{description}

\end{enumerate}

\vspace{0.2in} 

\begin{tabular}{ll}

$ \begin{array} {c||c|c|c|c}
        AB \bs CD & 00 & 01 & 11 & 10 \\ \hline \hline
        00        &    &    &    &    \\ \hline
        01        &    &    &    &    \\ \hline
        11        &    &    &    &    \\ \hline
        10        &    &    &    &    \\
\end{array} $

& 

$ \begin{array} {c||c|c|c|c}
        AB \bs CD & 00 & 01 & 11 & 10 \\ \hline \hline
        00        &    &    &    &    \\ \hline
        01        &    &    &    &    \\ \hline
        11        &    &    &    &    \\ \hline
        10        &    &    &    &    \\
\end{array} $

\\


\end{tabular}

\end{document}


